\documentclass[10pt]{article}

\usepackage[utf8]{inputenc}
\usepackage[T2A]{fontenc}
\usepackage[russian, english]{babel}

\usepackage{amssymb, amsmath, textcomp, tabularx, graphicx}
\usepackage{indentfirst}
\usepackage{listings}

\newtheorem{definition}{Определение}

\title{Проект 1.1}
\author{Андрей Коновалов, 073}
\date{}

\begin{document}

\maketitle

\section{Markov chains}

\subsection{Exercise 1.1}

Воспользуемся следующим определением марковской цепи с непрерывным временем.

\begin{definition}
  Однородная марковская цепь с непрерывным временем с интенсивностями переходов $\lambda(x, y)$ - это случайный процесс $X(t)$, который принимает значения из конечного или счетного пространства состояний $S$, и удовлетворяет
  \begin{gather*}
    P\{X(t + h) = x \; | \; X(t) = x\} = 1 - \lambda(x) h + o(h), \\
    P\{X(t + h) = y \; | \; X(t) = x\} = \lambda(x, y) h + o(h),
  \end{gather*}
  где $y \neq x$ и $\lambda(x) = \sum\limits_{y \neq x} \lambda(x, y)$.
\end{definition}

Процесс Пуассона $N$ с интенсивностью $\lambda > 0$ удовлетворяет
\begin{gather*}
  P\{N(t + h) = j \; | \; N(t) = j\} = 1 - \lambda h + o(h), \\
  P\{N(t + h) = j + 1 \; | \; N(t) = j\} = \lambda h + o(h).
\end{gather*}

А значит, по определению, процесс Пуассона является марковской цепью с непрерывным временем и пространством состояний $S = \{0, 1, 2, \ldots \}$.

\section{Walks on graphs}

\subsection{Exercise 2.1}

По сути, нам нужно показать эквивалентность следующих утверждений:
\begin{equation} \begin{aligned}
  \exists \; \pi'(x): \;\; \sum\limits_{x} \pi'(x) = 1, \\
  \forall y: \; \sum\limits_{x} \pi'(x) p(x, y) = \pi'(y)
\end{aligned} \end{equation}
и
\begin{equation} \begin{aligned}
  \exists \; \pi(x): \forall y: \; \sum\limits_{x} \pi(x) p(x, y) = \pi(y),  \\
  \sum\limits_{x, y} c(x, y) < 1, \;\; c(x, y) = \pi(x) p(x, y),
\end{aligned} \end{equation}
причем
\begin{equation} \begin{aligned}
  \pi(x) \sim \pi'(x),
\end{aligned} \end{equation}
предполагая, что соответствующий графу марковский процесс обратим.

\medskip

\subsubsection{(1) $\Rightarrow$ (2)}

Сначала покажем (1) $\Rightarrow$ (2).
\begin{gather*}
  (1) \; \Rightarrow \\
  \forall y: \; \sum\limits_{x} \pi'(x) p(x, y) = \pi'(y) \; \Rightarrow \\
  \sum\limits_{y} \sum\limits_{x} \pi'(x) p(x, y) = \sum\limits_{y} \pi'(y) = 1 \; \Rightarrow \\
  \sum\limits_{x, y} \frac{\pi'(x)}{2} p(x, y) = \frac{1}{2}.
\end{gather*}

Определим $\pi(x) = \frac{\pi'(x)}{2}$, тогда
\begin{gather*}
  \sum\limits_{x, y} \frac{\pi'(x)}{2} p(x, y) = \sum\limits_{x, y} \pi(x) p(x, y) = \sum\limits_{x, y} c(x, y) = \frac{1}{2} < 1.
\end{gather*}

А также
\begin{gather*}
  \forall y: \; \sum\limits_{x} \pi'(x) p(x, y) = \pi'(y) \; \Rightarrow \; \forall y: \; \sum\limits_{x} \pi(x) p(x, y) = \pi(y).
\end{gather*}

И, значит,
\begin{gather*}
  \exists \; \pi(x) = \frac{\pi'(x)}{2}: \forall y: \; \sum\limits_{x} \pi(x) p(x, y) = \pi(y),  \\
  \sum\limits_{x, y} c(x, y) < 1, \;\; c(x, y) = \pi(x) p(x, y),
\end{gather*}

\medskip

\subsubsection{(2) $\Rightarrow$ (1)}

Теперь покажем (2) $\Rightarrow$ (1).
\begin{gather*}
  (2) \; \Rightarrow \\
  \sum\limits_{x, y} c(x, y) = K < 1 < \infty.
\end{gather*}

Определим
\begin{gather*}
  \pi'(x) = \frac{\sum\limits_{y} c(x, y)}{\sum\limits_{x, y} c(x, y)} = \frac{\sum\limits_{y} c(x, y)}{K}.
\end{gather*}

Тогда
\begin{gather*}
  \sum\limits_{x} \pi'(x) = \sum\limits_{x} \frac{\sum\limits_{y} c(x, y)}{\sum\limits_{x, y} c(x, y)} = \frac{\sum\limits_{x, y} c(x, y)}{\sum\limits_{x, y} c(x, y)} = \frac{K}{K} = 1.
\end{gather*}

Заметим, что
\begin{gather*}
  \pi'(x) = \frac{\sum\limits_{y} c(x, y)}{K} = \frac{\sum\limits_{y} \pi(y) p(y, x)}{K} = \frac{\pi(x)}{K}.
\end{gather*}

Осталось показать, что
\begin{gather*}
  \forall y: \; \sum\limits_{x} \pi'(x) p(x, y) = \pi'(y).
\end{gather*}

Заметим, что $\forall y$
\begin{gather*}
  \sum\limits_{x} \pi'(x) p(x, y) = \pi'(y) \; \Leftarrow \\
  \sum\limits_{x} \frac{\sum\limits_{z} c(x, z)}{K} p(x, y) = \frac{\sum\limits_{x} c(y, x)}{K} \; \Leftarrow \\
  \sum\limits_{x} \sum\limits_{z} c(x, z) p(x, y) = \sum\limits_{x} c(y, x) \; \Leftarrow \\
  \forall x: \; \sum\limits_{z} c(x, z) p(x, y) = c(y, x) \; \Leftarrow \\
  \forall x: \; p(x, y) \sum\limits_{z} c(x, z) = c(y, x) \; \Leftarrow \\
  \forall x: \; p(x, y) \sum\limits_{z} c(x, z) = \pi(x) p(x, y) \; \Leftarrow \\
  \forall x: \; \sum\limits_{z} c(x, z) = \pi(x) \; \Leftarrow \\
  \forall x: \; \sum\limits_{z} \pi(z) p(z, x) = \pi(x) \; \Leftarrow \\
  (2).
\end{gather*}

И, значит,
\begin{gather*}
  \exists \; \pi'(x) = \frac{\pi(x)}{K}: \;\; \sum\limits_{x} \pi'(x) = 1, \\
  \forall y: \; \sum\limits_{x} \pi'(x) p(x, y) = \pi'(y).
\end{gather*}

\section{Queues}

\subsection{Exercise 3.1}

Опишем марковскую очередь с $\rho = \frac{\lambda}{\mu}$, где $\lambda$ - скорость появления новых клиентов, а $\mu$ - скорость обслуживания.

Каждое состояние марковского процесса будет описываться текущим количеством клиентов в очереди $k$.
Интенсивности переходов
\begin{gather*}
  q(k, k + 1) = \lambda, \;\;\; k = 0, 1, \ldots , \\
  q(k, k - 1) = \mu, \;\;\; k = 1, 2, \ldots .
\end{gather*}

\subsubsection{Случай $\rho < 1$}

В соответствии с упражнением 3.2, стационарное распределение вероятностей в этом случае
\begin{gather*}
  \pi(k) = (1 - \rho) \rho^k, \;\;\; k = 0, 1, \ldots .
\end{gather*}

Убедимся, что такая цепь является возвратной:
\begin{gather*}
  \pi(k) q(k, k + 1) = (1 - \rho) \rho^k \lambda, \\
  \pi(k + 1) q(k + 1, k) = (1 - \rho) \rho^{k + 1} \mu = (1 - \rho) \rho^k \lambda.
\end{gather*}

\subsubsection{Случай $\rho > 1$}

В соответствии с упражнением 3.2, стационарная мера в этом случае
\begin{gather*}
  \pi(k) = \rho^k, \;\;\; k = 0, 1, \ldots .
\end{gather*}

Убедимся, что такая цепь является возвратной:
\begin{gather*}
  \pi(k) q(k, k + 1) = \rho^k \lambda, \\
  \pi(k + 1) q(k + 1, k) = \rho^{k + 1} \mu = \rho^k \lambda.
\end{gather*}

\subsubsection{Случай $\rho = 1$}

В соответствии с упражнением 3.2, стационарная мера в этом случае
\begin{gather*}
  \pi(k) = 1, \;\;\; k = 0, 1, \ldots .
\end{gather*}

Убедимся, что такая цепь является возвратной:
\begin{gather*}
  \pi(k) q(k, k + 1) = \lambda, \\
  \pi(k + 1) q(k + 1, k) = \mu = \lambda.
\end{gather*}

В каждом из случаев цепь получилась возвратной, а каждая возвратная марковская цепь соответствует некоторому случайному блужданию на графе.

\subsection{Exercise 3.2}

Интенсивности переходов между состояниями определяются, как
\begin{gather*}
  q(k, k + 1) = \lambda, \;\;\; k = 0, 1, \ldots , \\
  q(k, k - 1) = \mu, \;\;\; k = 1, 2, \ldots .
\end{gather*}

Соответственно, матрица интенсивностей переходов определяется, как
\begin{gather*}
  Q = \left( \begin{matrix}
    -\lambda & \lambda          &                  &         & \\
    \mu      & -(\lambda + \mu) & \lambda          &         & \\
             & \mu              & -(\lambda + \mu) & \lambda & \\
             &                  &                  & \ddots  &  
 \end{matrix} \right).
\end{gather*}

Будем искать стационарное распределение в соответствии с основным кинетическим уравнением
\begin{gather*}
  \frac{d \pi}{d t} = \pi Q.
\end{gather*}

Расписав его построчно для нашей матрицы Q, получим
\begin{gather*}
  -\lambda \pi(0) + \mu \pi(1) = 0, \\
  \lambda \pi (k - 1) - (\lambda + \mu) \pi(k) + \mu \pi(k + 1) = 0, \;\;\; k \geq 1.
\end{gather*}

Можно легко убедиться, что эта система уравнений эквивалентна
\begin{gather*}
  \lambda \pi(k) = \mu \pi(k + 1), \;\;\; k \geq 0.
\end{gather*}

Таким образом, получаем
\begin{gather*}
  \pi(k) = \left( \frac{\lambda}{\mu} \right)^k \pi(0) = \rho^k \pi(0), \;\;\; k \geq 0.
\end{gather*}

Для случая $\rho < 1$ можно отнормировать
\begin{gather*}
  \sum\limits_k \pi(k) = \sum\limits_k \rho^k \pi(0) = \frac{1}{1 - \rho} \pi(0) = 1 \; \Rightarrow  \; \pi(0) = 1 - \rho
\end{gather*}
и получить стационарное рапределение вероятностей
\begin{gather*}
  \pi(k) = (1 - \rho) \rho^k, \;\;\; k \geq 0.
\end{gather*}

Для случая $\rho > 1$ можно взять $\pi(0) = 1$ и получить стационарную меру
\begin{gather*}
  \pi(k) = \rho^k, \;\;\; k \geq 0.
\end{gather*}

Так же и для случая $\rho = 1$ можно взять $\pi(0) = 1$ и получить стационарную меру
\begin{gather*}
  \pi(k) = 1, \;\;\; k \geq 0.
\end{gather*}

\end{document}
