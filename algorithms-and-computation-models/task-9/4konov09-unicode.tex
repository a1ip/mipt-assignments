\documentclass[10pt]{article}

\usepackage[cp1251]{inputenc}
\usepackage[T2A]{fontenc}
\usepackage[russian, english]{babel}

\usepackage{amssymb, amsmath, textcomp, tabularx, graphicx}

\newcolumntype{C}{>{\centering\arraybackslash}X}

\let \eps \varepsilon

\title{Задание 9}
\author{Коновалов Андрей, 074}
\date{}

\begin{document}

\maketitle

\noindent
\begin{tabularx}{\textwidth}{|C|C|C|C|C|C|C|C|C|}
  \hline
  1 & 2 & 3 & 4 & 5 & 6 & 7 & 8 & $\Sigma$ \\
  \hline
  &&&&&&&& \\
  \hline
\end{tabularx}

\bigskip

{\bf Задача 1}

Пусть есть КНФ, у которой в каждом дизъюнкте содержится не более трех литералов.
Для начала уберем повторяющиеся литералы, это делается за полиномиальное время.
Теперь КНФ состоит из дизъюнктов, которые содержат 1, 2 или 3 различных литералов.

Введя новые переменные заменим дизъюнкты, состоящие из 1 литерала следующим образом:
$$
  (p) \;\;\; \Rightarrow \;\;\; (p \vee q \vee r) \wedge (p \vee q \vee \neg r) \wedge
  (p \vee \neg q \vee r) \wedge (p \vee \neg q \vee \neg r)
$$
а дизъюнкты, состоящие из 2:
$$
  (p \vee q) \;\;\; \Rightarrow \;\;\; (p \vee q \vee r) \wedge (p \vee q \vee \neg r)
$$

Это делается за полиномиальное время.
Получившаяся КНФ будет иметь ровно 3 различных литерала в каждом дизъюнкте.

\medskip

{\bf Задача 2}

{\it (i)}
Протыкающее множество: $\{ x_1, x_2 \}$.

\smallskip

{\it (ii)}
Покольку $\chi$ имеет дизъюнкт $(\neg x_3)$, то протыкающее множесто должно содержать $\neg x_3$.
Протыкающее множество должно содержать хотя бы по одному элементу из $\{ x_1, \neg x_1 \}, \{ x_2, \neg x_2 \}$.
Протыкающее множество мощности 3 может содержать лишь по одному элементу из каждой пары, но при любой такой комбинации переменных оно не будет пересекать один из первых четырех дизъюнктов $\chi$.
Следовательно протыкающее множество должно иметь мощность больше 3.

\smallskip

{\it (iii)}
Дополним $\phi$ как в задаче 1:
$$
  \phi = (\neg x_1 \vee x_2 \vee x_3) \wedge (\neg x_1 \vee x_2 \vee \neg x_3)
$$

Теперь построим граф $G_{\phi}$, имеющий $2 n + 3 m = 2 \cdot 3 + 3 \cdot 2 = 12$.

\centerline{\includegraphics{{1}.png}}

Вершинное покрытие мощности $k = n + 2 m = 3 + 2 \cdot 2 = 7$ отмечено на рисунке большими кружками.

\medskip

{\bf Задача 3}

{\it (i)}
Нет, поскольку любые два элемента $M$ имеют хотя бы одну общую координату.

\end{document}
