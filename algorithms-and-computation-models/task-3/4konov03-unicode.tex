\documentclass[10pt]{article}

\usepackage[cp1251]{inputenc}
\usepackage[T2A]{fontenc}
\usepackage[russian, english]{babel}

\usepackage{amssymb, amsmath, textcomp, tabularx, graphicx}

\newcolumntype{C}{>{\centering\arraybackslash}X}

\let \eps \varepsilon

\title{Задание 3}
\author{Коновалов Андрей, 074}
\date{}

\begin{document}

\maketitle

\noindent
\begin{tabularx}{\textwidth}{|C|C|C|C|C|C|C|C|C|C|C|}
  \hline
  1 & 2 & 3 & 4 & 5 & 6 & 7 & 8 & 9 & 10 & $\Sigma$ \\
  \hline
  &&&&&&&&&& \\
  \hline
\end{tabularx}

\bigskip

Введем обозначения: $P$ - множество простых чисел, $a \; \% \; b$ - остаток от деления $a$ на $b$.

\medskip

{\bf Задача 1}

{\it (i)}
Вычисление $g_n$ самым "тупым" способом дает асимтотику $\Theta(n)$, т.к. необходимо сделать $n - 2$ шагов, вычисляя последовательно $g_3, g_4, ..., g_n$, при условии, что $g_1 = 3$ и $g_2 = 7$ (естественно, считая все по mod $19$).

\smallskip

{\it (ii)}
Докажем, что последовательность $\{ g_n \}$ периодична по любому модулю.
Пусть последовательность $\{ f_n = g_n \; \% \; 19 \}$.
Каждое следующее число в $f_n$ определяется только двумя предыдущими.
Заметим, что $\forall n \; 0 \leq f_n < 19$.
Поскольку $f_n$ бесконечна, а число возможных пар ее элементов конечно, то в какой-то момент $f_n$ зациклится.

Рассчитаем период для mod 19. Для $f_n$ получим
$3, 7, 17, 3, 4, 11, 7, 6, 0, 6,$
$12, 11, 15, 3, 2, 7, 16, 1, 18,$
$18, 16, 12, 2, 16, 15, 8, 12, 13,$
$0, 13, 7, 8, 4, 16, 17, 12, 3, 18,$
$1, 1, 3, 7$.
Последние 2 элемента $f_n$ совпали с первыми двумя, дальше все будет циклически повторяться.
Итого, период $f_n$ равен $40$.

Для вычисления $f_n$ в этом случае необходимо рассчитать период, который имеет размер $r = O(p^2)$, а затем за $\Theta(1)$ можно найти и $f_n$ взяв элемент $n \; \% \; r$ периода.
Итоговая асимптотика получается $O(p^2)$, или, если считать $p = const$, $\Theta(1)$.

\smallskip

{\it (iii)}
Явная формула для $g_n$ (была получена в предыдущем задании):
$$
  g_n = \frac{1}{2} \left( 1 + \sqrt{2} \right)^{n + 1} + \frac{1}{2} \left( 1 - \sqrt{2} \right)^{n + 1}
$$

Заметим, что $2$ является квадратичным вычетом по mod $23$, поскольку $x^2 \equiv 2 \pmod{23}$ при $x = 5$.
Из корректности выражения для $g_n$ следует, что все слагаемые, содержащие $\sqrt{2}$ уйдут, а значит, при подсчете $g_n$ по mod $23$ можно заменить $\sqrt{2}$ на $5$, так как $2 \equiv 5^2 \pmod{23}$.
Получим
$$
  g_n = \frac{6^{n + 1}}{2} + \frac{(-4)^{n + 1}}{2}
$$

Теперь можно считать $g_n$ считая числа вида $a^n$ за $\Theta(\log{n})$.

Посчитаем ответ для $n = 10000$.
Заметим, что по малой теореме Ферма $a^{22} \equiv 1 \pmod{23}$.
Поскольку $10001 \equiv 13 \pmod{22}$, то $a^{10001} \equiv a^{13} \pmod{23}$.
Получаем, что $g_{10000} \equiv \frac{6^{13} - 4^{13}}{2} \equiv 10 \pmod {23}$.  

\medskip

{\bf Задача 2}

{\it (i)}
Допустим, что больше чем для половины чисел из промежутка $1 \leq b < N$ выполнено $b^{N - 1} \equiv 1 \pmod{N}$.
Поскольку $a^{N - 1} \neq 1 \pmod{N}$, то для всех тех $b$ выполнено $(a b)^{N - 1} \neq 1 \pmod{N}$.

Заметим, что все эти $b$ различны. Иначе пусть $a b_1 \equiv a b_2 \pmod{N}$.
Вопользовавшись тем, что НОД($a$, $N$) = $1$ получаем, что $b_1 \equiv b_2 \pmod{N}$, а числа $b_1$ и $b_2$ были различны изначально.

Получаем, что каждому $b^{N - 1} \equiv 1 \pmod{N}$ соответствует число $(a b)^{N - 1} \neq 1 \pmod{N}$. Посколько чисел $b$ больше, чем половина промежутка $1 \leq b < N$, то и чисел $a b$ будет больше, чем половина, что невозможно.

\smallskip

{\it (ii)}
В тесте Ферма используются две ассимптотически "сложные" операции: подсчет НОД($a$, $N$) и $a^{N - 1} \pmod{N}$.
Возведение в степень $N$, как известно, выполняется за $\Theta(\log{N})$.

Алгоритм Евклида "дольше" всего выполняется на двух последовательных числах Фибоначчи.
Если $a = F_n$, $b = F_{n - 1}$, то будет выполнено $n - 2$ шагов алгоритма.
Учитвая, что числа Фибоначчи растут экспоненциально (как константа в степени $n$), получаем, что алгоритм Евклида выполняется за $O(\log{min(a, b)})$ шагов.
В нашем случае это $O(\log{N})$.

Получаем, что тест Ферма выполняется за $O(\log{N})$, а значит за полиномиальное по входу число операций.

\smallskip

{\it (iii)}
Тест Ферма не ошибается, если число составное, то есть, если он выдает ответ "нет".
Количество ответов "нет" не увеличится, если не делать проверку на НОД, но тогда, в соответствии с пунктом {\it (i)}, по крайней мере для половины чисел $a$ из промежутка $[0, N)$ выполнено $a^{N - 1} \neq 1 \pmod{N}$, а значит ответ для них будет "нет".
В итоге, по крайней мере для половины чисел ответ будет "нет", а значит правильный.

\medskip

{\bf Задача 3}

Что бы алгоритм был полиномиальным, необходимо, что бы его сложность была $O(polynom(\log{N}))$.
При выполнении решета Эратосфена нам необходимо "посетить" каждый элемент списка от $1$ до $N$ по крайней мере 1 раз.
Значит сложность алгоритма будет $\Omega(N)$.

При проходе по списку мы находим очередное невычеркнутое число и вычеркиваем кратные ему.
Понятно, что каждое найденное невычеркнутое число будет простым.
Получаем, что
$$
  complexity = \Theta \left( \frac{N}{2} + \frac{N}{3} + \frac{N}{5} + \frac{N}{7} + ... \right)
$$

Оценим сложность сверху
$$
  complexity = O \left( \frac{N}{2} + \frac{N}{3} + \frac{N}{4} + \frac{N}{5} + ... \right) = O(N \log{N})
$$

Получаем, что сложность решета Эратосфена $O(N \log{N})$.

\medskip

{\bf Задача 5}

Пусть открытый ключ Боба $(e, N) = (10, 899)$.
Тогда
$$
  N = p q \;\; \wedge \;\; p, q \in P \;\;\; \Rightarrow \;\;\; p = 29, \; q = 31
$$

Когда Боб вычислял $e$ он воспользовался неким числом $d$, таким, что
$$
  e \equiv d^{-1} \pmod{(p - 1) (q - 1)}
$$
или
$$
  10 d \equiv 1 \pmod{28 \cdot 30 = 840}
$$

Верно, что
$$
  10 d \equiv 1 \pmod{840} \;\;\; \Rightarrow \;\;\; 10 d \equiv 1 \pmod{10} 
$$
но
$$
  10 d \equiv 0 \pmod{10} 
$$

Значит такого числа $d$ не могло существовать и задача некорректна.

\medskip

{\bf Задача 7}

Воспользуемся тем, что если $n$ факторизовано как
$$
  n = p_1^{\alpha_1} \cdot ... \cdot p_k^{\alpha_k}
$$
то
$$
  \phi(n) = (p_1^{\alpha_1} - p_1^{\alpha_1 - 1}) \cdot ... \cdot (p_k^{\alpha_k} - p_k^{\alpha_k})
$$

Есть четыре "различных" способа "факторизовать" число $6$:
$$
  6 = 6 \cdot 1^t = 2 \cdot 3 = 2 \cdot 3 \cdot 1^t
$$
где каждый сомножитель соответствует одной скобке в разложении $\phi(n)$.

Решим уравнения
$$
  (p^k - p^{k - 1}) = v, \;\;\; p \in P, \; v \in \{ 1, 2, 3, 6 \}
$$

Получим решения
\begin{align*}
  v &= 1, \; k = 1, \; p = 2\\
  v &= 2, \; \{ k = 1, \; p = 3 \}, \; \{ k = 2, \; p = 2 \}\\
  v &= 3, \; \varnothing\\
  v &= 6, \; \{ k = 1, \; p = 7 \}, \; \{ k = 2, \; p = 3 \}\\
\end{align*}

Получаем, что существует 2 решения ($7$ и $3^2$), соответствующие первому способу факторизации и 2 решения($7 \cdot 2$ и $3^2 \cdot 2$), соответствующие второму.
Итого, $n \in \{ 7, 9, 14, 18 \}$.

\medskip

{\bf Задача 8}

Посчитаем показатель для каждого из чисел $(0, 19)$.
Получим

\smallskip

\noindent
\begin{tabularx}{\textwidth}{|C|C|C|C|C|C|C|C|C|C|C|C|C|C|C|C|C|C|}
  \hline
  1 & 2 & 3 & 4 & 5 & 6 & 7 & 8 & 9 & 10 & 11 & 12 & 13 & 14 & 15 & 16 & 17 & 18 \\
  \hline
  1 & 18 & 18 & 9 & 9 & 9 & 3 & 6 & 9 & 18 & 3 & 6 & 18 & 18 & 18 & 9 & 9 & 2 \\
  \hline
\end{tabularx}

\smallskip

{\it (i)}
Распределие по вычетов по показателям:

\smallskip

\noindent
\begin{tabularx}{\textwidth}{|C|C|C|C|C|C|}
  \hline
  1 & 2 & 3 & 6 & 9 & 18 \\
  \hline
  1 & 18 & 7, 11 & 8, 12 & 4, 5, 6, 9, 16, 17 & 2, 3, 10, 13, 14, 15 \\
  \hline
\end{tabularx}

\smallskip

{\it (ii)}
Первообразные корни: $2, 3, 10, 13, 14, 15$.

\end{document}
