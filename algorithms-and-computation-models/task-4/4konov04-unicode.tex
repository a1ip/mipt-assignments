\documentclass[10pt]{article}

\usepackage[cp1251]{inputenc}
\usepackage[T2A]{fontenc}
\usepackage[russian, english]{babel}

\usepackage{amssymb, amsmath, textcomp, tabularx, graphicx}

\newcolumntype{C}{>{\centering\arraybackslash}X}

\let \eps \varepsilon

\title{Задание 4}
\author{Коновалов Андрей, 074}
\date{}

\begin{document}

\maketitle

\noindent
\begin{tabularx}{\textwidth}{|C|C|C|C|C|C|C|C|C|C|}
  \hline
  1 & 2 & 3 & 4 & 5 & 6 & 7 & 8 & 9 & $\Sigma$ \\
  \hline
  &&&&&&&&& \\
  \hline
\end{tabularx}

\bigskip

{\bf Задача 1}

{\it (i)}
Решим рекуррентное соотношение
$$
  T(n) = T \left( \frac{n}{2} \right) + C n, \;\; T(1) = C, \;\; n = 2^k
$$

Проведем суммирование по дереву рекурсии
$$
  T(n) = \sum \limits_{i = 0}^k C \frac{n}{2^i} = \sum \limits_{i = 0}^k C 2^{k - i} = \sum \limits_{i = 0}^k C 2^{i} = C (2^{k + 1} - 1)
$$

Получаем ответ
$$
  T(2^k) = C (2^{k + 1} - 1)
$$

\medskip

{\bf Задача 2}

{\it (i)}
Что бы перемножить две матрицы размера $n$ алгоритмом I необходимо для каждого элементы матрицы результата сделать $n$ произведений и $n - 1$ сложение. Итоговое число арифметических операций:
$$
  T(n) = n^2 \cdot (n + (n - 1)) = 2 n^3 - n^2
$$

\smallskip

{\it (ii)}
Посчитаем отдельно число умножений $M(n)$, совершаемое во время перемножения двух матриц размера $n$ алгоритмом II. При перемножении матриц размера $n$ происходит $8$ перемножений матриц размера $\frac{n}{2}$. При $n = 1$ происходит просто перемножение чисел. Получаем
$$
  M(n) = 8 M \left( \frac{n}{2} \right), \;\; M(1) = 1
$$

Явное выражение будет
$$
  M(n) = M(2^k) = 2^{3k} = n^3
$$

Теперь отдельно посчитаем количество сложений $A(n)$ совершаемое во время перемножения двух матриц размера $n$. На каждом шаге алгоритма происходит $8$ перемножений матриц размера $\frac{n}{2}$ и $4$ сложения матриц того же размера. При $n = 1$ сложений не происходит. Получаем
$$
  A(n) = 8 A \left( \frac{n}{2} \right) + 4 \left( \frac{n}{2} \right)^2, \;\; A(1) = 0
$$

Суммированием по дереву рекурсии получаем
$$
  A(n) = \sum \limits_{i = 0}^{\log_2{n} - 1} 8^i \left( \frac{n}{2^i} \right)^2  + \sum A(1) = \sum \limits_{i = 0}^{\log_2{n} - 1} 2^i n^2
$$

Используя формулу для геометрической прогрессии получаем
$$
  A(n) = n^2 \frac{1 - 2^{\log_2{n}}}{1 - 2} = n^2 (n - 1)
$$

Суммарное число арифметический операций
$$
  T(n) = M(n) + A(n) = 2 n^3 - n^2
$$

\smallskip

{\it (iii)}
При выполнении шага алгоритма Карацубы происходит 7 умножений и 18 сложений матриц вдвое меньшего размера. Аналогично {\it (ii)} запишем рекуррентное соотношение для количества умножений
$$
  M(n) = 7 M \left( \frac{n}{2} \right), \;\; M(1) = 1
$$

Разрешая получаем
$$
  M(n) = 7^{\log_2{n}} = n^{\log_2 7}
$$

Для количества сложений
$$
  A(n) = 7 A \left( \frac{n}{2} \right) + 18 \left( \frac{n}{2} \right)^2, \;\; A(1) = 0
$$

Суммированием по дереву рекурсии получаем
$$
  A(n) = \sum \limits_{i = 0}^{\log_2{n} - 1} 18 \cdot 7^i \left( \frac{n}{2^{i + 1}} \right)^2  + \sum A(1) = \sum \limits_{i = 0}^{\log_2{n} - 1} \frac{18}{4} n^2 \left( \frac{7}{4} \right)^i
$$

Используя формулу для геометрической прогрессии получаем
$$
  A(n) = \frac{18}{4} n^2 \frac{1 - \left( \frac{7}{4} \right)^{\log_2{n}}}{1 - \frac{7}{4}} = 6 (n^2 - n^{\log_2{7}})
$$

Суммарное число арифметический операций
$$
  T(n) = M(n) + A(n) = 6 n^2 - 5 n^{\log_2{7}}
$$

\smallskip

{\it (iv)}
Выражение для количества сложений $A(n)$ уже было получено в {\it (ii)}
$$
  A(n) = n^2 (n - 1)
$$

\medskip

{\bf Задача 3}

{\it (i)}
Пусть А не верно. Тогда существует корректный алгоритм поиска минимального элемента в массиве из $n$ чисел, использующий только попарные сравнения, в котором существует реализуемый путь $W$ от корня к листу, имеющий меньше, чем $(n - 1)$ ребро.

Но, если верно B, то необходимо, что бы граф $G$ соответствующий пути $W$ был связен, а для этого необходимо, что бы он имел хотя бы $(n - 1)$ ребро. Поскольку путь должен $W$ имеет не меньше ребер, чем граф $G$ получаем противоречие.

\smallskip

{\it (ii)}
Допустим, что некоторый граф $G$ не связен, а значит разбивается на по крайней мере две компоненты связности. Возьмем любые две $M$ и $N$ из них. Выполним алгоритм дважды: в первом случае уменьшим все числа из компоненты $M$ на одно и то же число так, что бы минимум находился в $M$, во втором случае то же самое проделаем с числами из $N$, причем так, что бы минимум из $N$ не совпадал с минимумом из $M$ первого случая.

Результаты сравнений между числами из одной и той же компоненты связности не изменятся, а между компонентами сравнений не происходит. Соответственно в обоих случаях будет реализовываться тот путь, которому сопоставляется граф $G$. В обоих случаях алгоритм выдаст один и тот же ответ, а ответы должны быть разными. Противоречие.

\medskip

{\bf Задача 4}

Разобьем $\{ a_{f(1)}, ..., a_{f(n)} \}$ на группы $\{ b_1, ..., b_{\lceil \frac{n}{k} \rceil} \}$ по $k$ элементов. Назовем аналогичное разбиение отсортированного массива $\{ c_1, ..., c_{\lceil \frac{n}{k} \rceil} \}$. Заметим, что если число находится в группе $c_i$, то оно может находиться лишь в "соседних" группах $b_{i - 1}, b_{i}, b_{i + 1}$.

Теперь возьмем первые две группы $b_1, b_2$ и отсортируем их вместе за $O(2k \log{2k})$. Заметим, что первые $k$ элементов полученного массива будут составлять $c_1$. Оставшиеся назовем $d_2$.

Далее возьмем $d_2$ и $b_3$ и отсортируем их вместе за $O(2k \log{2k})$. Будем идти двумя указателями по массиву $c_1$ и полученному, сравнивая элементы, пропуская те, что входят в $c_1$ (при перемешивании они могли оказаться в $b_2$, а значит и в $d_2$). После пропуска останется $\geq k$ элементов и первые $k$ из них будут составлять $c_2$. Пропуск выполняется за $O(k)$. Оставшиеся назовем $d_3$.

Будем продолжать делать эту операцию и таким образом восстановим все группы $c_i$, а значит и отсортированный массив.

Оценим время выполнения
$$
  O(2k \log{2k}) + \left( \left\lceil \frac{n}{k} \right\rceil - 1 \right) (O(2k \log{2k}) + O(k)) = O(n \log{k})
$$
\end{document}
