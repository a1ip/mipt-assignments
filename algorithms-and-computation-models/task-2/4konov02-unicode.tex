\documentclass[10pt]{article}

\usepackage[cp1251]{inputenc}
\usepackage[T2A]{fontenc}
\usepackage[russian, english]{babel}

\usepackage{amssymb, amsmath, textcomp, tabularx, graphicx}

\newcolumntype{C}{>{\centering\arraybackslash}X}

\let \eps \varepsilon

\title{Задание 2}
\author{Коновалов Андрей, 074}
\date{}

\begin{document}

\maketitle

\noindent
\begin{tabularx}{\textwidth}{|C|C|C|C|C|C|}
  \hline
  1 & 2 & 3 & 4 & 5 & $\Sigma$ \\
  \hline
  &&&&& \\
  \hline
\end{tabularx}

\bigskip

{\bf Задача 1}
В соответствии с книгой Кормена ($I, 3.2$), для монотонно возрастающей функции $f$ верно
$$
  \int\limits_{m - 1}^n f(x) dx \leq \sum\limits_{k = m}^n f(k) \leq \int\limits_m^{n + 1} f(x) dx
$$

Гармонический ряд монотонно возрастает. Получим нижнюю оценку для гармонического ряда
$$
  \sum\limits_{k = 1}^n \frac{1}{k} \geq \int\limits_1^{n + 1} \frac{dx}{x} = \ln{(n + 1)}
$$

Получим верхнюю оценку (для ряда без первого члена)
$$
  \sum\limits_{k = 2}^n \frac{1}{k} \leq \int\limits_1^n \frac{dx}{x} = \ln{n}
$$

А значит
$$
  \ln{(n + 1)} \leq \sum\limits_{k = 1}^n \frac{1}{k} \leq \ln{n} + 1
$$

Откуда
$$
  \frac{\ln{(n + 1)}}{\ln n} \leq \frac{\sum\limits_{k = 1}^n \frac{1}{k}}{\ln n} \leq \frac{\ln{n} + 1}{\ln n}
$$

Переходя к пределу
$$
  1 \leq \lim\limits_{n \to \infty}{\frac{\sum\limits_{k = 1}^n \frac{1}{k}}{\ln n}} \leq 1
$$

Откуда
$$
  \sum\limits_{k = 1}^n \frac{1}{k} = \Theta(\ln{n})
$$

\medskip

{\bf Задача 3}

{\it (i)}
Обозначим $g_x(n)$ количество слов $\in G$ таких, что последняя буква слова есть $x$.
Тогда верно
$$
  g(n) = g_0(n) + g_1(n) + g_2(n)
$$

Поскольку после буквы $0$ может идти только буквы $0$ или $1$, а после $2$ - $1$ или $2$, то выполняется
$$
  g_0(n) = g_0(n - 1) + g_1(n - 1)
$$
$$
  g_2(n) = g_1(n - 1) + g_2(n - 1)
$$

После $1$ может идти любая буква, а значит
$$
  g_1(n) = g_0(n - 1) + g_1(n - 1) + g_2(n - 1) = g(n - 1)
$$

Получаем, что
$$
  g(n) = g_0(n - 1) + g_1(n - 1) + g(n - 1) + g_1(n - 1) + g_2(n - 1)
$$
$$
  g(n) = 2 \cdot g(n - 1) + g_1(n - 1) = 2 \cdot g(n - 1) + g(n - 2)
$$

Ручным подсчетом получаем, что
$$
  g(1) = 3, \;\;\; g(2) = 7
$$

Несколько следующих значений:
$$
  g(3) = 14, \;\;\; g(4) = 35
$$

\smallskip

{\it (v)}
Получим явное выражение для $g(n)$.
\begin{align*}
  g(n) &= 2 \cdot g(n - 1) + g(n - 2)\\
  g(n) &= c \cdot a^n\\
  c \cdot a^n &= 2 c \cdot a^{n - 1} + c \cdot a^{n - 2}
\end{align*}

Получим уравнение на $a$:
$$
  a^2 - 2 a - 1 = 0
$$

Решением будет
$$
  a = 1 \pm \sqrt{2}
$$

Значит $g(n)$ имеет вид
$$
  g(n) = c_1 (1 + \sqrt{2})^n + c_2 (1 - \sqrt{2})^n
$$

Подберем $c_1$ и $c_2$ так, что бы
$$
  g(1) = 3, \;\;\; g(2) = 7
$$

Получим
$$
  c_1 = \frac{1 + \sqrt{2}}{2}, \;\;\; c_2 = \frac{1 - \sqrt{2}}{2}
$$

Значит $g(n)$ имеет вид
$$
  g(n) = \frac{1}{2} (1 + \sqrt{2})^{n + 1} + \frac{1}{2} (1 - \sqrt{2})^{n + 1}
$$

Теперь можно вычислять $g(n)$ с помощью этой формулы, только я бы делал это не с конечной точностью, а с помощью быстрого возведения в степерь чисел $1 \pm \sqrt{2}$.

\medskip

{\bf Задача 4}

{\it (i)}
Будем делать аналогично книге Кормена ($I, 10.3$) для разбиения на "семерки" элементов.
Для количества чисел, которые будут заведомо больше "медианы медиан" получим
$$
  4 \left( \left\lceil \frac{1}{2} \left\lceil \frac{n}{7} \right\rceil \right\rceil - 2 \right) \geq \frac{2n}{7} - 8
$$

Значит алгоритм, рекурсивно вызываемый на пятом шаге, будет обрабатывать массив длиной не более $\frac{5n}{7} + 8$.

Пусть $T(n)$ - время работы алгоритма на массиве из $n$ элементов в худшем случае.
Тогда
$$
  T(n) \leq T \left( \left\lceil \frac{n}{5} \right\rceil \right) + T \left( \left\lfloor \frac{7n}{10} + 6 \right\rfloor \right) + O(n)
$$

\medskip

{\bf Задача 5}

{\it (i)}
В наихудшем случае нам всегда будет "доставаться" самый "далекий" от $k$ элемент, соответственно будет совершено $\Theta(n)$ шагов, на каждом из которых будет происходить престановка элементов за $\Theta(n)$. Итого в наихудшем случае будет $\Theta(n^2)$.

\end{document}
