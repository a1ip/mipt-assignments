\documentclass[10pt]{article}

\usepackage[cp1251]{inputenc}
\usepackage[T2A]{fontenc}
\usepackage[russian, english]{babel}

\usepackage{amssymb, amsmath, textcomp, tabularx, graphicx}

\newcolumntype{C}{>{\centering\arraybackslash}X}

\let \eps \varepsilon

\title{Задание 11}
\author{Коновалов Андрей, 074}
\date{}

\begin{document}

\maketitle

\noindent
\begin{tabularx}{\textwidth}{|C|C|C|C|C|C|}
  \hline
  1 & 2 & 3 & 4 & 5 & $\Sigma$ \\
  \hline
  &&&&& \\
  \hline
\end{tabularx}

\bigskip

{\bf Задача 1}

{\it (i)}
Утверждение не верно, например для 2-КНФ $(x \vee y) \wedge (x \vee \bar{y})$.
Она выполняется при $x = y = 1$, но в графе, ей соответствующем, есть путь из $\bar{x}$ в $x$.

\medskip

{\bf Задача 2}

{\it (i)}
В случае константы $c \leq \frac{1}{3}$ утверждение очевидно.
В случае константы $c \in (\frac{1}{3}, \frac{1}{2})$ нужно произвести $\log_c{\frac{1}{3}}$ проверок и выдать наиболее часто встречаемый из полученных ответ.
Полиномиальность по среднему числу шагов сохранится.

\medskip

{\bf Задача 5}

{\it (ii)}
При выполнении алгоритма $n - 2$ раза выбирается новое ребро и каждый раз, вероятность того, что оно входит в минимальный разрез не превышает $\frac{2}{|V_i|}$.
А значит вероятность, что ребро не входит в минимальный разрез не меньше, чем $1 - \frac{2}{|V_i|}$.

Получаем, что вероятность того, что полученный в итоге набор ребер будет минимальным разрезом не меньше, чем
$$
\frac{n - 2}{n} \cdot \frac{n - 3}{n - 1} \cdot \frac{n - 4}{n - 2} \cdot \frac{n - 5}{n - 3} \cdot ... \cdot \frac{5 - 2}{5} \cdot \frac{4 - 2}{4} \cdot \frac{3 - 2}{3}
$$

После сокращения получим
$$
\frac{1}{n} \cdot \frac{1}{n - 1} \cdot \frac{4 - 2}{1} \cdot \frac{3 - 2}{1} = \frac{2}{n (n - 1)}
$$

\smallskip

{\it (iii)}
Для того, что бы это доказать, достаточно показать, что функция
$$
  f(n) = \left( 1 - \frac{2}{n (n - 1)} \right)^{n^2}
$$
строго возрастает на $(2, + \infty)$, а так же, что
$$
  \lim_{n \to \infty} f(n) = e^{-2} < 0.15 = 1 - 0.85
$$

\end{document}
