\documentclass[10pt]{article}

\usepackage[cp1251]{inputenc}
\usepackage[T2A]{fontenc}
\usepackage[russian, english]{babel}

\usepackage{amssymb, amsmath, textcomp, tabularx, graphicx}

\newcolumntype{C}{>{\centering\arraybackslash}X}

\let \eps \varepsilon

\title{Задание 8}
\author{Коновалов Андрей, 074}
\date{}

\begin{document}

\maketitle

\noindent
\begin{tabularx}{\textwidth}{|C|C|C|C|C|C|C|C|C|}
  \hline
  1 & 2 & 3 & 4 & 5 & 6 & 7 & 8 & $\Sigma$ \\
  \hline
  &&&&&&&& \\
  \hline
\end{tabularx}

\bigskip

{\bf Задача 1}

{\it (i)}
Пронумеруем вершины сверху вниз слева направо.
Тогда для вершины $3$ утверждение A выполнено, но точкой раздела она не является.

\smallskip

{\it (ii)}
Утверждение B для вершины $3$ выполнено.

\smallskip

{\it (iii)}
Вычислим $low(v)$, получим
$$
  low(1) = low(2) = 1; \;\;\; low(3) = low(4) = \varnothing;
$$

\medskip

{\bf Задача 3}

{\it (i)}
Необходимо найти путь с максимальным $\Pi_{i = 1}^{k - 1} r(v_i, v_{i + 1}) = L$.
Заметим, что максимизировать $L$ в случае положительных ребер, это все равно, что максимизировать $\log{L}$, поскольку он строго монотонно возрастает.

Если в графе есть нулевые ребра, то для начала можно просто поиском в глубину проверить наличие пути с их использованием, а затем выкинуть из графа.
Если теперь мы найдем положительный путь, то будем использовать его, если нет - то уже найденный нулевой (если он был, конечно).

Теперь заметим, что $\log{L} = \log{r_1} + ... + \log{r_{k - 1}}$.
Соответственно необходимо промаксимизировать эту сумму.
Поскольку $r_i \leq 1$, то $\log{r_i} < 0$.
Получаем, что промаксимизировать ту сумму, это все равно что пронимизировать сумму $|\log{r_1}| + ... + |\log{r_{k - 1}}|$.
Теперь заменим метки на ребрах $r_i$ на $|\log{r_i}|$ и воспользуемся алгоритмом Дейкстры, что бы найти путь минимальной длины, а соответственно и с минимальной суммой меток.

При решении мы используем поиск в глубину и алгоритм Дейкстры, а значит сложность будет $O(|V|^2)$.

\end{document}
