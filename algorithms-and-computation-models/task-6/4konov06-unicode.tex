\documentclass[10pt]{article}

\usepackage[cp1251]{inputenc}
\usepackage[T2A]{fontenc}
\usepackage[russian, english]{babel}

\usepackage{amssymb, amsmath, textcomp, tabularx, graphicx}

\newcolumntype{C}{>{\centering\arraybackslash}X}

\let \eps \varepsilon

\title{Задание 5}
\author{Коновалов Андрей, 074}
\date{}

\begin{document}

\maketitle

\noindent
\begin{tabularx}{\textwidth}{|C|C|C|C|C|C|}
  \hline
  1 & 2 & 3 & 4 & 5 & $\Sigma$ \\
  \hline
  &&&&& \\
  \hline
\end{tabularx}

\bigskip

{\bf Задача 4}

{\it (i)}
Вычисления алгоритма Гаусса:
$$
  \begin{pmatrix}
    1 & 0 & 0 & 5\\
    8 & 1 & 0 & 25\\
    4 & 8 & 1 & 125
  \end{pmatrix}
  \rightarrow
  \begin{pmatrix}
    1 & 0 & 0 & 5\\
    0 & 1 & 0 & -15\\
    0 & 8 & 1 & 5
  \end{pmatrix}
  \rightarrow
  \begin{pmatrix}
    1 & 0 & 0 & 5\\
    0 & 1 & 0 & -15\\
    0 & 0 & 1 & 125
  \end{pmatrix}
$$

Поскольку знаменатели всех чисел, встречающихся при вычислении равны 1, то я их опустил.

{\it (ii)}
Проверим:
$$
  1 = det(D^{(3)}) = d_{33}^{(2)} = d_1 \cdot d_2 \cdot a_{33}^{(2)} = 1 \cdot 1 \cdot 1 = 1
$$

\end{document}
