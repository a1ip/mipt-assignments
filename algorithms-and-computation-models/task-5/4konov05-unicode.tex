\documentclass[10pt]{article}

\usepackage[cp1251]{inputenc}
\usepackage[T2A]{fontenc}
\usepackage[russian, english]{babel}

\usepackage{amssymb, amsmath, textcomp, tabularx, graphicx}

\newcolumntype{C}{>{\centering\arraybackslash}X}

\let \eps \varepsilon

\title{Задание 5}
\author{Коновалов Андрей, 074}
\date{}

\begin{document}

\maketitle

\noindent
\begin{tabularx}{\textwidth}{|C|C|C|C|C|C|C|C|C|C|}
  \hline
  1 & 2 & 3 & 4 & 5 & 6 & 7 & 8 & 9 & $\Sigma$ \\
  \hline
  &&&&&&&&& \\
  \hline
\end{tabularx}

\bigskip

{\bf Задача 1}

{\it (i)}
Рекурсивно вычислим БПФ массивов A и B.

\smallskip

\noindent
\begin{tabularx}{\textwidth}{|C|C|C|C|C|C|C|C|C|}
  \hline
  & $w^0_8$ & $w^1_8$ & $w^2_8$ & $w^3_8$ & $w^4_8$ & $w^5_8$ & $w^6_8$ & $w^7_8$ \\
  \hline
  $A(w_8^0)$ & 6 & 0 & 0 & 0 & 0 & 0 & 0 & 0\\
  \hline
  $A(w_8^1)$ & 1 & 0 & 3 & 2 & 0 & 0 & 0 & 0\\
  \hline
  $A(w_8^2)$ & 1 & 0 & 0 & 0 & 3 & 0 & 2 & 0\\
  \hline
  $A(w_8^3)$ & 1 & 2 & 0 & 0 & 0 & 0 & 3 & 0\\
  \hline
  $A(w_8^4)$ & 4 & 0 & 0 & 0 & 2 & 0 & 0 & 0\\
  \hline
  $A(w_8^5)$ & 1 & 0 & 3 & 0 & 0 & 0 & 0 & 2\\
  \hline
  $A(w_8^6)$ & 1 & 0 & 2 & 0 & 3 & 0 & 0 & 0\\
  \hline
  $A(w_8^7)$ & 1 & 0 & 0 & 0 & 0 & 2 & 3 & 0\\
  \hline
\end{tabularx}

\smallskip

\noindent
\begin{tabularx}{\textwidth}{|C|C|C|C|C|C|C|C|C|}
  \hline
  & $w^0_8$ & $w^1_8$ & $w^2_8$ & $w^3_8$ & $w^4_8$ & $w^5_8$ & $w^6_8$ & $w^7_8$ \\
  \hline
  $B(w_8^0)$ & 8 & 0 & 0 & 0 & 0 & 0 & 0 & 0\\
  \hline
  $B(w_8^1)$ & 3 & 3 & 0 & 2 & 0 & 0 & 0 & 0\\
  \hline
  $B(w_8^2)$ & 3 & 0 & 3 & 0 & 0 & 0 & 2 & 0\\
  \hline
  $B(w_8^3)$ & 3 & 2 & 0 & 3 & 0 & 0 & 0 & 0\\
  \hline
  $B(w_8^4)$ & 3 & 0 & 0 & 0 & 5 & 0 & 0 & 0\\
  \hline
  $B(w_8^5)$ & 3 & 0 & 0 & 0 & 0 & 3 & 0 & 2\\
  \hline
  $B(w_8^6)$ & 3 & 0 & 2 & 0 & 0 & 0 & 3 & 0\\
  \hline
  $B(w_8^7)$ & 3 & 0 & 0 & 0 & 0 & 2 & 0 & 3\\
  \hline
\end{tabularx}

\smallskip

{\it (ii)}
Для С получим

\smallskip

\noindent
\begin{tabularx}{\textwidth}{|C|C|C|C|C|C|C|C|C|}
  \hline
  & $w^0_8$ & $w^1_8$ & $w^2_8$ & $w^3_8$ & $w^4_8$ & $w^5_8$ & $w^6_8$ & $w^7_8$ \\
  \hline
  $C(w_8^0)$ & 48 & 0 & 0 & 0 & 0 & 0 & 0 & 0\\
  \hline
  $C(w_8^1)$ & 3 & 3 & 9 & 17 & 6 & 6 & 4 & 0\\
  \hline
  $C(w_8^2)$ & 9 & 0 & 9 & 0 & 13 & 0 & 17 & 0\\
  \hline
  $C(w_8^3)$ & 3 & 17 & 4 & 3 & 6 & 0 & 9 & 6\\
  \hline
  $C(w_8^4)$ & 22 & 0 & 0 & 0 & 26 & 0 & 0 & 0\\
  \hline
  $C(w_8^5)$ & 3 & 6 & 9 & 0 & 6 & 3 & 4 & 17\\
  \hline
  $C(w_8^6)$ & 9 & 0 & 17 & 0 & 13 & 0 & 9 & 0\\
  \hline
  $C(w_8^7)$ & 3 & 0 & 4 & 6 & 6 & 17 & 9 & 3\\
  \hline
\end{tabularx}

\smallskip

{\it (iii)}
Умножив матрицу на полученный вектор, убедимся, что получается $[3, 3, 9, 17, 6, 6, 4, 0]$.

\medskip

{\bf Задача 3}

{\it (i)}
Убедимся, что $3$ - примитивный корень степени $8$ в простом поле $Z_{41}$.

\smallskip

\noindent
\begin{tabularx}{\textwidth}{|C|C|C|C|C|C|C|C|C|C|}
  \hline
  $n$ & 0 & 1 & 2 & 3 & 4 & 5 & 6 & 7 & 8 \\
  \hline
  $3^n$ & 1 & 3 & 9 & 27 & 40 & 38 & 32 & 14 & 1\\
  \hline
\end{tabularx}

\end{document}
