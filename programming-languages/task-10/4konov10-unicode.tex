\documentclass[10pt]{article}

\usepackage[cp1251]{inputenc}
\usepackage[T2A]{fontenc}
\usepackage[russian, english]{babel}

\usepackage{amssymb, amsmath, textcomp, tabularx, graphicx}

\newcolumntype{C}{>{\centering\arraybackslash}X}%

\let \eps \varepsilon

\title{Задание 10}
\author{Коновалов Андрей, 074}
\date{}

\begin{document}

\maketitle

\noindent
\begin{tabularx}{\textwidth}{|C|C|C|C|C|C|C|C|}
  \hline
  1 & 2 & 3 & 4 & 5 & 6 & 7 & $\sigma$ \\
  \hline
  &&&&&&& \\
  \hline
\end{tabularx}

\bigskip

{\bf Задача 2}

$\varphi^{-1}(1)$ это те слова $\Sigma^*$, которые при морфизме $\varphi$ переходят в 1.
Слово $\eps$ является единицей моноида $\Sigma^*$, а значит оно переходит в 1.
Все слова вида $a^+$ переходят в 1.
Слова, содержащие хотя бы одну букву $b$ переходят в 0.
Итого, $\varphi^{-1}(1) = a^*$.

\medskip

{\bf Задача 3}

Построим по исходному алфавиту $\Sigma$, моноиду $M$ и множеству $X \subseteq M$ конечный автомат $A$, принимающий язык $L = \varphi^{-1}(X)$.

Состояниями $A$ будут элементы $M$.
Начальное состояние соответствует единице $M$, финальные - элементам $X$.
Для каждого состояния $q$, соответствующего элементу $m$, переход по букве $a$ будет осуществляться в состояние, соответствующее элементу $m \cdot \varphi(a)$.

Докажем $L \subseteq L(A)$.
$\forall w \in L \rightarrow \varphi(w) \in X$.
После обработки $w$, $A$ окажется в состоянии, соответствующем $\varphi(w[0]) \cdot ... \cdot \varphi(w[-1]) = \varphi(w) \in X$.
Состояние, соотвествующее $\varphi(w)$ - финальное, а значит $w$ будет принято.

Докажем $L(A) \subseteq L$.
После обработки любого $w \in L(A)$, $A$ окажется в финальном состоянии, соответствующем $\varphi(w)$.
Следовательно $\varphi(w) \in X$, а значит $w \in L$.

\medskip

{\bf Задача 4}

Пользуемся представлением элементов $M$ ввиде двудольных графов.
Включим в множество $X$ каждый элемент $M$, такой, что в графе, ему соответствующем, есть ребро из вершины левой доли, сооветствующей начальному состоянию $A$, в вершину правой доли, соотвествующей какому-то финальному состоянию $A$.

Докажем $\varphi^{-1}(X) \subseteq L(A)$.
Возьмем любое $w \in \varphi^{-1}(X)$.
Заметим, что $\varphi(w) \in X$.
Значит, что в графе, соответствующем $\varphi(w)$, есть ребро из вершины левой доли, соответствующей начальному состоянию $A$, в вершину правой доли, соответствующей какому-то финальному состоянию $A$.
Получаем, что после обработки слова $w$, $A$ окажется в финальном состоянии и слово будет принято.

Докажем $L(A) \subseteq \varphi^{-1}(X)$.
Возьмем любое $w \in L(A)$.
Заметим, что $\varphi(w)$ будет соответствовать граф, в котором есть ребро из вершины левой доли, сооветствующей начальному состоянию $A$, в вершину правой доли, соотвествующей какому-то финальному состоянию $A$.
Значит мы включили $\varphi(w)$ в $X$.
Получаем, что $w \in \varphi^{-1}(X)$.

\medskip

{\bf Задача 5}

{\bf (i)}
$M_2$ является замыканием множества $\{ 123, 223, 133 \}$ ($123 \rightarrow$ для краткости опускается), относительно операции "умножения".

Получаем, что $M_2 = \{ 123, 223, 133, 233, 333 \}$, дальнейшее умножение элементов между собой не приводит в появлению новых.

Построим таблицу умножения:

\begin{center}
\begin{tabular}{|c||c|c|c|c|c|}
  \hline
  & 123 & 223 & 133 & 233 & 333 \\
  \hline
  \hline
  123 & 123 & 223 & 133 & 233 & 333 \\
  \hline
  223 & 223 & 223 & 333 & 333 & 333 \\
  \hline
  133 & 133 & 233 & 133 & 233 & 333 \\
  \hline
  233 & 233 & 233 & 333 & 333 & 333 \\
  \hline
  333 & 333 & 333 & 333 & 333 & 333 \\
  \hline
\end{tabular}
\end{center}

\smallskip

{\bf (ii)}
Воспользуемся задачей 4. Единственным элементом, у графа которого есть переход из $1$ левой доли в $3$ правой доли является $333$.

Получаем, что $X = \{ 333 \}$.

\medskip

{\bf Задача 6}

{\bf (i)}
Класс эквивалентности отношения конгруенции, содержащий элемент $x$ обозначим $[x]$.
Определим операцию умножения: $[x] \cdot [y] \rightarrow [xy]$.

Докажем корректность введенной операции умножения, то есть независимость результата от выбора представителей умножаемых классов.

Другими словами, докажем, что

$$x_1 \sim y_1 \wedge x_2 \sim y_2 \rightarrow x_1 x_2 \sim y_1 y_2$$  

Допустим, что

$$\neg (x_1 x_2 \sim y_1 y_2) \Leftrightarrow \exists w_0, z_0 ( w_0 x_1 x_2 z_0 \in L \wedge w_0 y_1 y_2 z_0 \notin L )$$

Распишем $x_1 \sim y_1$, как

$$\forall w, z ((w x_1 z \in L \wedge w y_1 z \in L) \vee (w x_1 z \notin L \wedge w y_1 z \notin L))$$

Взяв $w = w_0$, $z = x_2 z_0$, и используя, что $w x_1 z \in L$, получаем, что

$$w y_1 z = w_0 y_1 x_2 z_0 \in L$$

Аналогично из $x_2 \sim y_2$ и $w_0 y_1 y_2 z_0 \notin L$, получаем:

$$w_0 y_1 x_2 z_0 \notin L$$

Приходим к противоречию, значит

$$x_1 x_2 \sim y_1 y_2$$

Корректность операции умножения доказана.

Ассоциативность умножения следует из ассоциативности операции конкатенации, а единицей будет элемент $[\eps]$. Получаем, что множество классов эквивалентности отношения конгруенции является моноидом.

\smallskip

{\bf (ii)}
Что бы такая функция была морфизмом, необходимо, что бы операция сохранялась.
Покажем это:

$$\varphi(x y) = [x y] = [x] [y] = \varphi(x) \varphi(y)$$

\end{document}
