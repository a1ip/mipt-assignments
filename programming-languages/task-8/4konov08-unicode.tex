\documentclass[10pt]{article}

\usepackage[utf8]{inputenc}
\usepackage[T2A]{fontenc}
\usepackage[russian, english]{babel}

\usepackage{amssymb, amsmath, textcomp, tabularx, graphicx}

\newcolumntype{C}{>{\centering\arraybackslash}X}%

\title{Задание 7}
\author{Коновалов Андрей, 074}
\date{}

\let \eps \varepsilon

\begin{document}

\maketitle

\noindent
\begin{tabularx}{\textwidth}{|C|C|C|C|C|C|C|C|}
  \hline
  0 & 1 & 2 & 3 & 4 & 5 & 6 & $\sigma$ \\
  \hline
  &&&&&&& \\
  \hline
\end{tabularx}

\bigskip

{\bf Задача 1}

Построим множества $t_{ij}$ для $i = 1 \, .. \, 6$, $j = 1 \, .. \, 6 - i + 1$ для слова $ababab$ в соответствии с алгоритмом. Получим:

\noindent
\begin{tabularx}{\textwidth}{|C|C|C|C|C|C|C|}
  \hline
  $t_{ij}$ & 1 & 2 & 3 & 4 & 5 & 6\\
  \hline
  1 & $\{ A \}$ & $\{ S \}$ & $\{ A \}$ & $\{ S \}$ & $\{ A \}$ & $\{ S \}$\\
  \hline
  2 & $\{ S \}$ & $\{ A \}$ & $\{ S \}$ & $\{ A \}$ & $\{ S \}$ &\\
  \hline
  3 & $\{ A \}$ & $\{ S \}$ & $\{ A \}$ & $\{ S \}$ &&\\
  \hline
  4 & $\{ S \}$ & $\{ A \}$ & $\{ S \}$ &&&\\
  \hline
  5 & $\{ A \}$ & $\{ S \}$ &&&&\\
  \hline
  6 & $\{ S \}$ &&&&&\\
  \hline
\end{tabularx}

Вертикальная нумерация соответствует $j$, горизонтальная - $i$.

Поскольку $S \in t_{16}$, то $ababab \in L$.

\medskip

{\bf Задача 3}

{\bf (i)}
Покажем, что лемма о разрастании для линейных языков не выполняется.
Для $\forall k \geq 0$, возьмем слово $z = b^k a^k b^k \in L_2$, и посмотрим на произвольное его разбиение $z = uvwxy$, такое, что $|uvxy| \leq k$, $|vx| > 0$.

$|z| = 3k \wedge |uvxy| \leq k \Rightarrow |w| \geq 2k$. В таком случае, получаем, что подслово $a^k$ слова $z$ полностью лежит в $w$. А значит $v$ и $x$ имеют вид $b^t, t \geq 0$. Поскольку $|vx| > 0$, то либо $|v| > 0$, либо $|x| > 0$, а значит в слове $p = u v^i w x^i y$ при $i = 2$ не будет выполняться соотношение $|p|_b = 2 |p|_a$, а значит $p \notin L_2$.

Поскольку лемма о разрастании не выполняется, то $L_2$ - нелинейный.

\smallskip

{\bf (ii)}
Нет, не согласен. Любое из слов вида $b^k a^k b^k$, где $k \geq 3$ не подпадает ни под один из перечисленных шаблонов (перечисленных в фигурных скобках в условии), и, при этом, счетчик не обнулится внутри этого слова, значит рассуждение не верно.
Тем не менее, исходное утверждение верно. Его доказательство приведено мной в задаче 5 задания 6.

\smallskip

{\bf (iii)}
Докажем, что $M$ принимает $L_2$. Введем следующую "потенциальную функцию":
в состоянии $q_0$ собираются слова $w$ для которых $|w|_b = 2 |w|_a$, причем в стеке находится лишь $Z$;
в состоянии $q_{+}$ собираются слова $w$ для которых $|w|_b - 2 |w|_a > 0$, причем количество символов $1$ в стеке равно $|w|_b - 2 |w|_a$;
в состоянии $q_{-}$ собираются слова $w$ для которых $|w|_b - 2 |w|_a < 0$, причем количество символов $1$ в стеке равно $||w|_b - 2 |w|_a|$.

Проверим корректность всех переходов.

Проверим переходы из $q_0$.
При переходе $(a, Z, 11Z)$ в состояние $q_{-}$ число $|w|_b - 2 |w|_a = -2 < 0$, причем в стеке окажется 2 символа $1$.
При переходе $(b, Z, 1Z)$ в состояние $q_{+}$ число $|w|_b - 2 |w|_a = 1 > 0$, причем в стеке окажется 1 символ $1$.

Проверим переходы из $q_{+}$.
При переходе по $b$ счетчик увеличивается на 1, а в стек, соответственно дописывается 1 символ $1$.
При переходе по $a$ возможны несколько случаев: счетчик = 1, тогда осуществляется переход в $q_{-}$, а счетчик становится равным -1, в стеке оказывается 1 символ 1; счетчик = 2, тогда осуществляется переход в $q_{0}$, а счетчик становится равным 0, в стеке не оказывается символов $1$; счетчик > 2, тогда в стеке лежит $\geq 3$ символов 1, а значит возможен и осуществляется переход $(a, 111, 1)$ в $q_{+}$. При всех переходах "потенцальная функция" выполняется для текущего считанного слова. 

Аналогично проверяются переходы из $q_{-}$.

\smallskip

{\bf (vi)}
Да, поскольку для него выполняются условия в определении: $(i)$ выполняется, $(ii)$ выполняется, $(iii)$ очевидно выполняется, поскольку $\eps$-переходов вообще в автомате нет.

\medskip

{\bf Задача 4}

{\bf (i)}
Покажем, что лемма о разрастании для КС-языков не выполняется.
Для $\forall k \geq 0$, возьмем слово $w = a^k b^k c^k \in L$, и посмотрим на произвольное его разбиение $w = uvzxy$, такое, что $|vzx| \leq k$, $|vx| > 0$.

$|vzx| \leq k$. В таком случае $vzx$ не может содержать буквы $a$ и $c$ одновременно, а значит либо $a$, либо $c$ не будет содержаться ни в одном слове из $v$ и $x$. Получаем, что в слове $p = u v^i w x^i y$ при $i = 2$ не будет выполняться соотношение $|p|_a = |p|_b = |p|_c$, а значит $p \notin L$.

Поскольку лемма о разрастании не выполняется, то $L_2$ - не КС.

\smallskip

{\bf (ii)}
Язык $\bar{L} = \{ w \; | \;|w|_a \neq |w|_b \vee |w|_b \neq |w|_c \vee |w|_c \neq |w|_a \}$. Докажем, что язык КС, построив F-автомат, который его принимает. Для построения сначала построим 3 автомата, которые принимают языки $\{ |w|_a \neq |w|_b \}$, $\{ |w|_b \neq |w|_c \}$, $\{ |w|_c \neq |w|_a \}$ используя конструкцию счетчика с маркером дна стека $z$. Теперь построим новый автомат, в котором будет новое начальное состояние, соединенное со всеми бывшими начальными состояними переходами вида $(\eps, z, z)$. Построенный автомат будет принимать все слова, принимаемые хотя бы одним из построенных до этого автоматов и никакие другие, а значит и язык $\bar{L}$.

\end{document}
