\documentclass[10pt,twocolumn]{article}
\usepackage[cp1251]{inputenc}
\usepackage[T2A]{fontenc}
\usepackage[russian,english]{babel}
\textheight=250mm
\textwidth=186mm
\hoffset=-8mm
\voffset=-28mm
\pagestyle{empty}
\usepackage{amssymb}
%\title{ ЗАДАНИЕ 1: ИНДУКЦИЯ, ЯЗЫКИ ДИКА (СКОБОЧНЫХ ВЫРАЖЕНИЙ)}
%\title{
%\author{}
\let\empty\varnothing
\let\ph\varphi
\let\eps\varepsilon
\def\T{{\cal T}}
\def\L{{\cal L}}
\def\F{{\cal F}}
\def\Q{{\cal Q}}
\def\N{{\cal N}}
\def\A{{\cal A}}
\def\B{{\cal B}}
\def\TT{\tilde{\cal T}}
\def\f{\tilde f}
\def\ind{\mathop{\rm index}}
\def\St{\mathop{\rm St}}
\let\bd\partial
\def\V{\ensuremath{{\cal V}}}
\def\SS{{\mathbb S}}
\def\RR{\mathbb R}
\def\R{\cal R}
\def\ZZ{\mathbb Z}
\def\s{\Sigma }
\def\ss{\Sigma^* }
\def\ra{\rightarrow}
\def\Ra{\Rightarrow}
\renewcommand{\figurename}{Рис.}

%\def\theenumi{\roman{enumi}}

\newtheorem{Definition}{Определение}
 % \newtheorem{Lem}{Лемма}
  \newtheorem{Rk}{Замечание}
 \newtheorem{Quest}{Вопрос}
 \newtheorem{Answer}{Ответ}
\renewcommand \theDefinition {}
\renewcommand \theRk {}

\begin{document}
%\maketitle
\centerline{ЗАДАНИЕ 1: ЯЗЫКИ, МОРФИЗМЫ}
\centerline{КОНЕЧНЫЕ АВТОМАТЫ}
\centerline{{\bf срок сдачи 20:00  9 сентября 2011}}

\centerline{{\bf Проходной балл: 0.4}}

{\footnotesize$ $
Литература: 
 Ахо А., Ульман Д. 
{\it Теория Синтаксического Анализа, Перевода и Компиляции}  М.: Мир, 1978, Гл. 0, 2.

Серебряков В.А., Галочкин М.П., Гончар Д.Р., Фуругян М.Г.
{\em Теория и реализация языков программирования.} М.: МЗ-пресс, 2003.

Хопкрофт Д., Мотвани Р., Ульман Д. 
{\em Введение в теорию автоматов, языков и вычислений.} М.,: Вильямс, 2002


{\em КЛЮЧЕВЫЕ СЛОВА (минимальный необходимый объем понятий и навыков по
этому разделу): принцип мат. индукции, язык, дополнительный язык,
операции над языками, морфизм,
обратный морфизм;  
%, порождающая грамматика,  КС- и автоматные грамматики;
конечные автоматы (КА), детерминированные, полные и недетерминированные КА, $\eps$-такты; регулярные языки,
%алгебра регулярных выражений,  примеры нерегулярных языков;
поиск подстрок, алгоритм Кнута- Морисса- Пратта.

%языки скобочных выражений (языки Дика). 
}
}

\smallskip


{\footnotesize Необходимо набрать $\geq 0.4$ балла. Если что-то не
получается,-- не смущайтесь, лучше  сдать работу с одной задачей, чем
не сдавать ничего.

Начиная с 2008 года изменился формат нашего курса:

-- курс делится на две части;

-- на лекциях будет проведено две короткие контрольные (их невыполнение влечет получение дополнительных задач на экзамене);

-- по итогам первой половины в начале  ноября  проводится двухчасовой 
письменный экзамен (неуспевающие в течение семестра получают 
дополнительные задачи).
%вечером того же дня к 17.00 объявляются итоги;

-- успешно сдавшие экзамен получают оценку в зачетку и могут прослушать вторую половину,
которая считается курсом по выбору и оценивается в декабре дифференцированным зачетом; 
эта часть является необходимой для продолжения обучения
по специализации "Компьютерные технологии". 

В нашем случае перед каждым тестом (экзаменом или --- для продолжающих обучение --- зачетом) 
формируется {\em предварительная оценка}.
От нее зависит число дополнительных задач, которые вы рискуете получить на экзамене\footnote{Кстати, 
лично мне такой способ учета неуспеваемости (дополнительные задания на экзамене) кажется 
справедливым. Он, конечно, требует дополнительных усилий и реально увеличивает трудность экзамена. 
Основной вывод (который пока звучит как страшилка): стандартный способ подготовки к 
экзамену за 5 минут до (или после) его начала
может оказаться весьма неэффективным.}.

Предварительная оценка для первой части курса вычисляется по следующей формуле
(положительная оценка и отсутствие долгов по лекциям являются основанием для получения автомата):
%Студенты, работавшие в течение семестра,
%могут рассчитывать на получение досрочной оценки, вычисляемой по следующей %формуле.
$32$ \%
оценки  составляют баллы, полученные за
выполнение $8$ домашних заданий (каждое задание может дать $4$
\%).  $68$ \% оценки составляют контрольные, {\bf которые будут
проводиться на каждом семинаре} (соответственно, каждая контрольная
может принести $8.5$ \%).   Формула вычисления оценки:
$\lceil P/20 \rceil$, где $ P= 32\%
\frac{1}{8}\sum_{i=1}^{8}{\frac{сумма\; баллов\; за\; i-е\;
задание}{проходной\; балл\; i-о\; задания}} +68\%
\frac{1}{8}\sum_{i=1}^{8}{\frac{сумма\; баллов\; за\; i-ю\;
контр.}{проходной\; балл\; i-й\; контр.}}$

Положительная оценка ($\geq 3$) засчитывается в качестве экзаменационной. От этой оценки можно отказаться и
пойти сдавать экзамен с ``нуля''.

{\bf Регулярно выполняющие недельные задания студенты, освобождаются
от сдачи ``канонического задания''. Наоборот, должники, для того,
чтобы быть допущенным к контрольным и к экзамену, будут обязаны выполнить и сдать
каноническое задание (баллы за сдачу канонического задания не
начисляются). Кроме того, должники имеют наименьший приоритет во всех
очередях. Данные о текущей успеваемости каждого студента будут
помещаться в сеть.

Дополнительные задачи на экзамене получают студенты, не получившие положительную оценку и не сдавшие (указанную часть) канонического задания.
}

После номера каждой задачи в скобках указывается балл. Мои субъективные
расценки таковы: $\leq 0.25$ балла--- простые задачи; $0.5$ балла и выше---
задачи средней трудности, требующие определенной техники; $1$ балл и выше--- сложные задачи.

Хочу обратить внимание на то, что, как правило, суммарный балл всех
задач в задании, тесте или контрольной {\bf существенно больше
проходного} (особенно это касается заданий). Но отсюда не вытекает, что получить полный балл просто.
Кроме того, повышают оценку полученные бонусы (за
решение тяжелых задач или за нетривиальные аргументы).
Как показывает практика, основные дискуссии будут проходят по двум
тесно связанным между собой темам:  что считать очевидным и что считать
доказанным. Формальные объекты, изучаемые в курсе (регулярный язык,
контекстно- свободный язык, порождающая грамматика, однозначная
грамматика, конечный автомат, автомат со стеком --- простые и естественные, но требуют
обязательного {\em самостоятельного} продумывания, и обращаться с ними
(по крайней мере на первых порах) нужно аккуратно, т.е.
проводить полные доказательства по индукции или от противного (а других
пока не изобрели,--- фактически наш курс изучает индуктивные объекты,
которые зачастую могут вести себя достаточно неожиданно.)

Еще одно важное замечание. В недельные задания наряду с задачами по
пройденным темам включены задачи по темам {\em следующего} семинара,
а также отдельные, более тяжелые, бонусные задачи, предназначенные для
более глубокого изучения материала.
Кроме того, некоторые темы будут излагаться только в заданиях, и вам,
возможно, придется поискать их в книгах или в сети. Для облегчения
вашей работы я постараюсь, насколько это возможно, приводить в
заданиях теоретический материал, и тут мне безусловно нужна обратная
связь.


Приведу ПРИМЕРНУЮ раскладку первой части курса по семинарам:
%\footnote{Возможны изменения после
%согласований сроков контрольных
%с моими коллегами.}:

3.09. Вводный семинар.
Общие свойства языков. 
%Морфизмы.
Конечные автоматы. 
%Поиск подстрок.
Контр. $1$
%Языки скобочных выражений.
%(языками Дика).

10.09. Конечные автоматы и регулярные языки $1$.
Алгебра регулярных выражений.
Различные формы представления регулярных языков (регулярные грамматики, НКА, ДКА). 
Контр. $2$.  Сдается $1$-е недельное задание.


17.09. Конечные автоматы и регулярные языки $2$.
Теорема эквивалентности.
Преобразования между
разными представлениями.
Теорема о разрастании для
регулярных языков.
Контр.
$3$.  Сдается $2$-е недельное задание.

24.09. Конечные автоматы и регулярные языки $3$.
 Уравнения с регулярными коэффициентами. 
Рациональные и распознаваемые ряды и их эквивалентность (теорема 
Шютценберже).
Свойства замкнутости регулярных языков. 
 Контр. $4$.  Сдается $3$-е недельное задание.


1.10. Конечные автоматы и регулярные языки $4$.
Отношения, определяемые конечными автоматами. Моноиды.  Синтаксический моноид. Построение минимальных
автоматов. Контр. $5$.  Сдается $4$-е недельное задание.



8.10.  
Порождающие грамматики. Контекстно- свободные грамматики (КСГ).
Однозначные КСГ.
Контр. $6$.
Сдается $5$-е недельное задание.


15.10. Стандартные формы КС-языков. КС-языки и  автоматы со стеком.
Контр. $7$.
Сдается $6$-е недельное задание.

22.10 Свойства замкнутости КС- языков. Лемма о разрастании для КС-
языков. 
Контр. $8$.
Сдается $7$-е недельное задание.

29.10. Детерминированные  автоматы со стеком и однозначные КС-грамматики.
Сдается $8$-е недельное задание.

5.11. Повторение пройденного материала и подготовка к экзамену.
Информация о предварительных оценках за курс.

??.11. Письменный экзамен.


\smallskip

Задание состоит из короткой части (собственно, ОБЯЗАТЕЛЬНОГО задания) и 
НЕОБЯЗАТЕЛЬНОГО набора практических упражнений по теме. По моему замыслу, студенты, 
выполняющие дополнительные задачи, должны получить определенные практические
преимущества, поскольку часть задач в текущих тестах и контрольных будет
аналогична задачам из дополнительного списка. Однако, основная цель включения
дополнительных задач,--- это возможность самостоятельного изучения предмета.

{\bf Пожалуйста, {\large ВНИМАТЕЛЬНО} прочитайте идущие ниже правила оформления заданий!}

{\bf 1.} Задание должно быть выполнено в одном из следующих форматов 
TEX, DVI, PS, PDF, RTF, WORD($\leq$2003). 
%(в последнем случае ВМЕСТЕ с RTF-файлом можно передать
%соответствующий WORD-файл; если я смогу решить проблему диалектов WORD'a и прочитать ваш файл,
%то в дальнейшем вы можете выполнять задания в этом формате).

{\huge {\bf
Не используйте, пожалуйста, WORD-2007 и OPEN WORD.}
}

{\bf Возможно, нам следует еще более сузить список допустимых текстовых редакторов.
Например, договориться, чтобы использовать ТОЛЬКО Open Office и установить его из одного дистрибутива.
С TEX'ом проблем меньше, но я попрошу ВСЕХ, выполняющих задание в TEX'е, видимо, под LINUX, 
и использующих кодировку utf8 посылать мне исходник в кодировке windows. Также полезно 
к исходному TEX-файлу послать соответствующий откомпилированный файл в формате PDF или PS. 
DVI-файлы, к сожалению,
только по названию являются device-independent и не всегда корректно обрабатываются разными версиями 
TEX'а. Опять же можно договориться об использовании выделенного дистрибутива TEXа.}

{\bf 2.} Файл с выполненным заданием должен называться 
[последняя цифра номера вашей группы][первые 5 букв латинского написания вашей фамилии]
[N задания (2 цифры)].[расширение файла tex, dvi, ps, pdf,
doc,rtf]

Например, если студент Иванов из группы  074 выполнит $7$-е задание в формате RTF, то файл должен
называться: 

4ivano07.rtf

{\bf 3.} Задание отсылается  электронной почтой до 20:00 пятницы текущей недели   по адресу:

algzadanie@gmail.com

{\bf При этом файлы с заданием должны быть 
быть ЗАПАКОВАНЫ и присоединены к письму  в виде ПРИЛОЖЕНИЯ. 

}

Например, уже упомянутый студент Иванов, должен запаковать файл 6ivano07.rtf, допустим, используя
архиватор ZIP, и затем отправить полученный файл 6ivano07.zip  в виде приложения.

Если же Иванов использовал TEX и выполненное задание содержится в файлах 
6ivano07.tex и picture1.bmp, picture2.bmp, $\dots$ (где BMP-файлы содержат картинки), то
ВСЕ файлы должны быть запакованы в архив 6ivano07.zip, который посылается в виде ПРИЛОЖЕНИЯ.
{\bf ПОЖАЛУЙСТА, не нужно НИКАКИХ изысков в виде самораспаковывающихся архивов и т.д.}

{\huge Размер архива не должен превышать 150K !!!}


{\bf 4.} 
На  первой странице (или титульном листе) должна быть ваша фамилия и
номер группы, а также нарисована табличка  (n-- число задач в текущем
задании) для учета выполненных заданий
\begin{tabular}{|c|c|c|c|c|c|}
\hline
1&2&3&\dots&n&$\sigma$\\
\hline
&&&\dots&&\\
\hline
\end{tabular}

{\em Отсутствие таблицы стоит $-0.5$ балла.}

{\bf 5.} 
Я проверяю задания и рассылаю их вам, ЕСЛИ они выполненны в формате TEX, WORD или RTF. При этом 
выполняющие задания в TEXе получат проверенные
задания с формулами. Выполняющие задания в формате WORD или RTF могут рассчитывать только на текстовые
замечания. 

Для сведения. Пакет TEX, разработанный Д.Кнутом, является неофициальным 
стандартом для написания текстов ``с формулами'', и, добавлю от себя, 
легко воспринимается людьми, привыкшими иметь дело с формулами. Эстетические
качества документов, выполненных на TEXе, очень высоки. Для чуть более опытных просто
скажу, что на TEXе можно готовить оригинал-макеты книг, 
{\em удовлетворяющих всем типографским требованиям}. Поэтому знание TEXа
не представляется мне лишним для физтехов.

Электронную версия очень хорошей книги С.Львовского о TEXе можно найти на сайте:

ftp://www1.mccme.ru/pub/tex/lvovsky-newbook/llang.ps

Вопрос о дистрибутивах TEXа, конечно, нелишний (мне известны
два пакета EMTEX и MIKTEX). 
Я могу принести дистрибутивный CD с TEXом, прилагаемым к книге С.Львовского, но мне говорили, что
он есть в вашей локальной сети. На этом CD есть, в частности, стилевой пакет 
russkor, где имеются различные изыски для людей с повышенными эстетическими 
запросами, например, специальные русские кавычки. 

Для удобства я поместил в сеть TEX-файл, в кодировке windows и соответствующий PS-файл $1$ задания.
Определенную трудность для начинающих представляют кириллические фонты. Короткий совет:
если вы работаете в MIKTEX, который был установлен с CD, прилагаемого к книге С.Львовского,
то используйте DOS-кодировку исходного TEX-файла и команду rlatex (этот вариант использую я сам),
В оригинальном пакете MIKTEX, нет такой команды, но можно использовать WIN-кодировку, команду latex и пакет babel, для этого в преамбуле
должны стоять декларации:

\verb|\usepackage[cp1251]{inputenc}|

\verb|\usepackage[T2A]{fontenc}|

\verb|\usepackage[english,russian]{babel}|

Приведу рекомендации по работе в TEXе, написанные вашим коллегой
(стиль изложения сохранен).

{\bf 1. Где взять TEX?}  

Для работы необходимы: \\
- MIKTEX можно взять на \verb|/hltdistrofficeTEX| \\
- очень удобный редактор для ТЕХа --- WinEDt --- можно взять там же, или на http://winedt.com \\
Не испольуйте кряки\footnote{Я сам не понимаю, что означает этот термин,
но из контекста становится ясно, что регистрировать редактор не рекомендуется.}, которые с ней предлагаются. 
ОНИ НЕ РАБОТАЮТ!!! Точнее, работают, но через пару минут редактор прекращает работу без сохранения. 
Лучше месяц использовать  незарегистрированный редактор, а потом реинсталлировать его.

{\bf 2. Что нужно сделать с заданием, чтобы оно откомпилировалось? }

Если установить MIKTEX и winedt, открыть задание в winedt и откомпилировать его, то  получится DVI-файл, в котором ничего русского нет,
поэтому рекомендуется поступать следующим образом.


- Перевести ТЕХ-файл в кодировку windows-1251,
используя для этого программу Akelpad. В ней все очень просто. Открываете файл, в меню выбираете: 
Кодировки -> Открыть как DOS-866. Потом - Кодировки -> Сохранить в WINDOWS-1251. ВСЁ. \\
- Поменять в WINedt кодировку шрифта на кириллическую. 
Для этого запускаем WINedt В меню Options -> preferences 
Находим панель font там нажимаем кнопку font И внизу где написан набор символов 
ставим кириллический или cyrillic.\\
- Пишем в начале на следующей строке после "documentclass" три заветные строчки 

\verb|\usepackage[cp1251]{inputenc}|

\verb|\usepackage[T2A]{fontenc}|

\verb|\usepackage[english,russian]{babel}|

Далее нажимаем (в winEDt) ``Ctrl+Shift+x'' и должно всё заработать. 

Теперь замечание от меня, повторюсь: если вы работаете в TEXе и используете кодировку 
utf8, посылайте, пожалуйста, исходних в кодировке windows или DOS.

Если даже после исчерпывающих рекомендаций трудности возникнут,
воспользуйтесь старинной программистской поговоркой: если ничего не помогает,--- прочти инструкцию.
На первых порах,  вы можете прислать мне даже
неоткомпилированные файлы $1$-о задания,--- я постараюсь их просмотреть и исправить возможные синтаксические
ошибки.

{\bf Важное замечание. Конечно, оформление задания в электронном виде требует определенных усилий, но я считаю, что 
умение грамотно изложить свои мысли с ФОРМУЛАМИ и КАРТИНКАМИ входит в ликбез. На первых порах неумение оформлять текст
(в частности, неумение набирать формулы) будет прощаться, но в дальнейшем тексты, скажем, с некорректным изображением формул просто не будут проверяться. Мне кажется,
что на весь процесс обучения может быть отведено не более трех первых заданий.}

При выполнении заданий, пожалуйста, не гонитесь за количеством (я имею
в виду переписывание решений из максимального числа источников), а
немного подумайте над новыми понятиями. 

Обсуждение заданий с товарищами не возбраняется. Необходимо
только УКАЗАТЬ заимствования в тексте, во-первых, по-моему так принято
в приличном обществе, а, во-вторых, всегда полезно понять, что знаешь
или не знаешь ты сам. Если заимствование без указания на источники
будет замечено, то  {\huge ВСЕ} ПОДЕЛЬНИКИ БУДУТ ОШТРАФОВАНЫ.
{\em Я прошу серьезно отнестись к последнему замечанию, поскольку  задания будут 
выполняться в электронном виде, где как операция заимствования, команда COPY,
так и идентификация заимствования не требуют специальных усилий, поэтому,
пожалуйста, не подводите своих товарищей.
Я отдаю себе отчет в том, что принцип коллективной ответственности несколько несправедливый, но не 
вижу другого выхода, поскольку ограничения могут иметь только моральный характер.}


Я бы всячески приветствовал обратную связь: замечания, комментарии,
предложения и т.д., но последние несколько лет ее практически не было, например, на форуме нашего
 сайте не было обсуждений. Возможно, публика привыкла работать в формате новомодный социальных сетей.
Можно обсудить и такую возможность.

Адрес моей электронной почты: starasov@newmail.ru

{\large Не отправляйте, пожалуйста,  задания по ЭТОМУ АДРЕСУ!}



PS-файлы и TEX-файлы с текущими заданиями, а также результаты и сообщения
будут размещены на нашем сайте сайте: algzadanie.do.am
(PS-файлы можно смотреть с помощью программы ghostview, которую можно
скачать с сайта www.cs.wisc.edu/$\sim$~ghost).

Кроме того, возникающие вопросы будут обсуждаться на форуме нашего сайте
algzadanie.do.am
(Отдельная благодарность двум вашим товарищам, создавшим этот сайт.)

Более осмысленные замечания я смогу сделать после проверки работ.}

\bigskip




\centerline{\bf Слова, языки, операции над языками} 
\centerline{\bf (Гомо)морфизмы}

\medskip

{\footnotesize
{\em Обозначения.} Пусть $\s$--- некоторый конечный алфавит. Тогда
$\ss$--- множество всех слов в $\s$. Пустое слово обозначается
$\eps$. Языком называется любое подмножество $\ss$. Пустой язык
(не содержащий {\em никаких} слов) обозначается $\emptyset$.

Говорят, что слово {\em содержит квадрат, куб,$\dots\, k$-ю степень}, если 
оно содержит, соответственно, подслово $ss$, $sss,\dots, \underbrace{ss\dots s}_{k \,\mbox{слагаемых}}$, где
$s\neq\eps$. Например, слово $u=01010$ содержит квадрат.  Слово имеет {\em нахлест (overlap)}, если оно 
содержит подслово вида $awawa$, где $a$ --- литерал (буква), а $w$ --- некоторое слово. Например, слово
$u$ имеет нахлест. Бесконечное слово $w=a_1a_2\dots$ называется {\em асимптотически периодическим}, если 
существуют натуральные $n, p$, что при всех $i>n$ $a_i=a_{i+p}$.

С языками можно производить любые теоретико-множественные операции.
Кроме того, вводится операция {\em конкатенации (или со\-е\-ди\-нения)}. По
определению, если $L_1, L_2 \subseteq \ss$, то их соединение $L_1 \cdot
L_2$ состоит из всех слов вида $uv, \; u \in L_1, v \in L_2$.
{\em Итерация} языка $L$ обозначается $L^*$ и, по определению, равна:
$L^* \stackrel{def}{=} \eps + L + L\cdot L + L\cdot L\cdot L + \dots$.
Можно также ввести операции {\em правого} и {\em левого} деления.
{\em Правое частное} языков определяется так: $L_1
\swarrow L_2 \stackrel{\mbox{def}}{=}\{x \, | \, \exists y \in L_2: \,
xy \in L_1\}$,  т.е. множество префиксов слов из $L_1$, чьи суффиксы
принадлежат $L_2$.
{\em Левое частное} языков определяется так: $L_1
\searrow L_2 \stackrel{\mbox{def}}{=}\{y \, | \, \exists x \in L_2: \,
xy \in L_1\}$,  т.е. множество суффиксов слов из $L_1$, чьи префиксы
принадлежат $L_2$.

Декартово произведение (языков) обозначается $\times$.

Мощность множества $L$ обозначается $|L|$.

Множество всех подмножеств множества $L$ обозначается
$2^L$.

Метасимвол $?$ обозначает любой символ из $\s$.

Число букв $a$ в слове $x$ обозначается $|x|_a$.

{\em (Гомо)морфизмом}\footnote{В современной литературе вместо 
понятия {\em ``гомоморфизм''} все чаще употребляется термин {\em``морфизм''}, который мы и будем использовать} называется любое отображение $\s_1 \ra \ss_2$,
где $\s_1, \s_2$--- алфавиты (т.е. каждому литералу $\s_1$ ставится в
соответствие слово $\s_2$).  Можно считать, что область определения
морфизма есть $\ss_1$, полагая $h(\eps)=\eps$ и индуктивно
$h(xa)=h(x)h(a)$ для всех $x \in \ss_1, \; a \in \s_1$. Тогда, по
определению, образ $h(L)$ языка $L \subseteq \ss_1$ состоит из цепочек
$\{h(w)\, | \, w \in L\}$. 
Таким образом, морфизм --- это естественное {\bf АЛФАВИТНОЕ КОДИРОВАНИЕ} (вообще говоря, не взаимно однозначное!) 
языка $L \subseteq \ss_1$ 
в алфавите $\s_2$ посредством
кодирования всех литералов 
алфавита $\s_1$ некоторыми словами в алфавите $\s_2$.
Также, по определению, для любого $y \in
\ss_2$ {\em обращение морфизма} $h^{-1}(y)$ состоит из всех слов в
алфавите $\s_1$, которые отображаются морфизмом $h(\cdot)$ в $y$,
т.е. $h^{-1}(y)=\{x \,|\, h(x)=y\}$. Аналогично для  $L \subseteq
\ss_2$ определяется язык $h^{-1}(L)=\cup_{y \in L} h^{-1}(y)=\{x \in \ss_1 \,|\,
h(x) \in L \subseteq \ss_2\}$.

Слово $w$ называется {\em (бесконечной) неподвижной точкой} морфизма 
$h : \ss\to\ss$,
если его можно получить итерируя морфизм, т.е. существует такой символ 
$a\in \s$, что для всех $i=1,2\dots$ слово 
$h^{(i)}(a)\stackrel{def}=\overbrace{h(h(\dots h(a)\dots ))}^{i \mbox{ раз}}$
является префиксом $w$ или символически: $w=\lim_{i\to \infty} h^{(i)}(a)$.
Неподвижными точками морфизмов являются многие известные 
последовательности, например, знаменитая, {\em не содержащая кубов} 
двоичная последовательность
Хедлунда-Морса $X=\lim_{i\to\infty}\mu^{(n)}$, задающаяся морфизмом
$\mu(0)=01, \mu(1)=10$ (см. доп. задачу N 4). 


}


\medskip

{\bf 1 ($0.1$ балла.)}
Определим язык $L\subseteq \{a,b\}^*$ индуктивными правилами:
({\em 1}) $\eps \in L$; ({\em 2}) вместе с любым словом $x \in L$ в $L$ также 
входят слова $bax, baax, bbax, bbaax$; ({\em 3}) никаких других слов в $L$ нет.

В язык $T\subseteq \{a,b\}^*$ входит пустое слово $\eps$ и ВСЕ 
начинающиеся с $b$ и заканчивающиеся $a$ слова, в которых нет подслов
``$aaa$'' или ``$bbb$'' (в словах нет трех одинаковых символов подряд). 
Докажите или опровергните, что $L=T$. 

{\footnotesize Если равенство неверно, то нужно явно указать слово, принадлежащее
одному языку и не принадлежащее другому. Если равенство верно, то нужно провести
доказательство ПО ИНДУКЦИИ: ({\em i}) $L\subseteq T$;
({\em ii}) $T\subseteq L$ (т.е. $L=T$).}

\smallskip


{\bf 2
($3 \times 0.02$ балла.)}
\noindent ({\em i})
$\{ a^{3n} | \, n>0\} \cap \{ a^{5n+1} | n \geq 0\}^* =?$

\noindent ({\em ii}) $\emptyset \cap \{\eps\} =?$

\noindent ({\em iii})
Определим морфизм на $\s^*=\{a,b\}^*$: $h(a)=aaa, \, h(b)=aa$.
Пусть $L=\{a^{7n} \, | \, n\geq 0\}$. 
$h^{-1}(L)=?$

\smallskip

{\bf 3 ($2 \times 0.04$ балла).} 
Пусть $L \subseteq \ss$ --- произвольный
язык, а $h(\cdot): \ss \ra \ss$ --- произвольный морфизм. Верно ли,
что

\noindent ({\em i})  $h (h^{-1}(L)) = L$?

\noindent ({\em ii}) $h^{-1}(h(L)) = L$?

({\em Обоснованием является доказательство, если
соотвествующее равенство справедливо, и построение контр\-примера,если
оно ложно.})


\medskip

\centerline{\bf РЕГУЛЯРНЫЕ ЯЗЫКИ И}
%КОНЕЧНЫЕ АВТОМАТЫ}
\centerline{\bf И КОНЕЧНЫЕ АВТОМАТЫ I}


{\footnotesize$ $
Литература: 
 Ахо А., Ульман Д. 
{\it Теория Синтаксического Анализа, Перевода и Компиляции}  М.: Мир, 1978, \S \S 2.2-2.3.

%Серебряков В.А., Галочкин М.П., Гончар Д.Р., Фуругян М.Г.
%{\em Теория и реализация языков программирования.} М.: МЗ-пресс, 2003.

Хопкрофт Д., Мотвани Р., Ульман Д. 
{\em Введение в теорию автоматов, языков и вычислений.} М.,: Вильямс, 2002, Гл. 1-4
%(осторожно, в переводе много неточностей!!!). 


\medskip


\begin{Definition}[Конечный автомат]
(Не\-де\-тер\-ми\-ни\-ро\-ван\-ным ко\-неч\-ным) автоматом 
называется пятерка $A = (Q, \s, \delta, q_0, F)$, где

%\begin{enumerate}
\noindent {\em ({\em i})} $Q$ --- конечное множество {\em состояний} управляющего устрой\-ст\-ва (УУ);
\noindent {\em ({\em ii})} $\s$ --- входной алфавит;
\noindent {\em ({\em iii})}  $\delta$ --- отображение $Q \times \s$ в множество всех подмножеств
$2^Q$, называемое функцией переходов УУ;
\noindent {\em ({\em iv})} $q_0$ --- начальное состояние УУ;
\noindent {\em ({\em v})} $F \subseteq Q$ --- множество финальных состояний.
%\end{enumerate}
\end{Definition}

На вход УУ подается слово $w=a_1a_2\ldots a_n$, записанное на {\em
входной ленте}. Автомат обрабатывает его слева направо по {\em
тактам}. Если автомат находится в текущем состоянии $q$ и
читает текущий символ $a$ входного слова $w$, то такт состоит из
(недетерминированной)
смены состояния по формуле $q_{new}  \in \delta(q, a)$ и перехода к
следующему символу $a_{i+1}$ на входной ленте. 

{\em Автомат с выходом}-- это конечный
автомат, который дополнительно на каждом переходе $(q_{old}$,
$[$прочитанный символ $]) \ra (q_{new},$ [~выходной
символ$])$ выдает также выходной символ из некоторого фиксированного
выходного алфавита. Если некоторое слово принимается автоматом, то
соответствующее выходное слово, по определению, принадлежит {\em выходному
языку}. 

По определению, на каждом такте КА может читать не более одного символа, а если это КА с выходом, то запрещено
выдавать более одного символа. Такие автоматы называются
{\bf синхронными}, в отличие от {\bf асинхронных}, которые могут за один такт прочитать и$/$или выдать несколько
символов. Эквивалентное название асинхронных КА с выходом {\bf конечные (рациональные) преобразователи [rational
transductions]}. 
%Мы поговорим о них в следующих заданиях.

Упомянем также две модификации
определения КА. В {\em КА с  $\eps$-тактами} состояние может изменяться {\em без чтения входного символа},
т.е. $\delta(\cdot,\cdot):$ $Q \times (\s\cup\eps) \to 2^Q$.
У ``двусторонних'' ({\em two-way}) конечных автоматов УУ может двигаться по входной ленте вправо и влево, 
{\em не выходя за пределы входного слова и не изменяя его}, т.е. 
$\delta(\cdot,\cdot):$ $Q \times \s \to 2^Q\times \{\mbox{право, лево}\}$. 


Назовем {\em конфигурацией} пару $(q,w) \in Q \times \s^*$.
На множестве конфигураций введем соответствующее тактам работы
автомата бинарное отношение $\vdash$:  для всех $q'
\in \delta(q,a)$ и для всех $w \in \s^* \, (q,aw) \vdash (q',w)$.
Рефлексивное и транзитивное замыкание отношения $\vdash$ обозначим
$\vdash^*$

\begin{Definition}[Автоматный язык]
Автомат $A$ {\em принимает (рас\-по\-знает, допускает)} слово $w \in \s^*$,
если $(q_0,w) \vdash^* (q,\eps), \, q \in F$.

Автоматный язык $L(A)$ состоит из всех слов, принимаемых
автоматом $A$, т.е. $L(A) = \{w|w \in \s^*, \, (q_0,w) \vdash^*
(q,\eps)$ для неко\-то\-ро\-го $q \in F\}$.

Автоматы $A_1$ и $A_2$ называются эквивалентными, если при\-ни\-маемые ими
языки совпадают $L(A_1)=L(A_2)$
\end{Definition}

Языки, принимаемые КА, называются  \textit{регулярными}
или \textit{рациональными}.

КА удобно изображать графически посредством т.н. {\em диаграммы Мура}.
Построим на множестве состояний $Q$, как на вершинах,
ориенти\-ро\-ванный помеченный {\em граф (переходов)} $\Gamma$ (возможно
имеющий петли и кратные дуги). Правило построения дуг такое: для любой
тройки $(q,a,q')$ такой, что $q' \in \delta(q,a)$, проводится
ориентированная дуга от $q$ к $q'$ и помечается символом $a$.
Примеры диаграмм приведены в задаче N 1 и доп. задачах {NN 1, 4}.

{\large {\bf Я бы рекомендовал и попросил бы {\em в обязательном порядке} сопровождать построение КА его диаграммой,
поскольку по графу гораздо эффективнее проверять корректность построения (даже при минимальных навыках).}}


Сопоставим любому ориентированному пути в графе $\Gamma$
слово на дугах. Тогда предыдущее определение говорит, что слово $w$
принимается автоматом $A$, если в графе $\Gamma(A)$ существует
ориентированный путь, начинающийся в вершине $q_0$ и  за\-канчивающийся в
одной финальных вершин, который сопостав\-лен\footnote{Подчеркнем,
что финальные вершины не являются {\em погло\-ща\-ющими}. Требуется только,
чтобы ориентированный путь, отвеча\-ю\-щий допустимому слову, начинался в
$q_0$ и {\em заканчивался} в какой-то финальной вершине.} $w$.
Таким образом, КА можно использовать для представления путей в помеченном орграфе и, как мы увидим, многие 
манипуляции с автоматами имеют прямые аналогии с алгоритмами теории графов.

\begin{Quest}\label{not-bb}
Постройте автоматы и их диаграммы, принимающие следующие языки в алфавите $\s=\{a,b\}$

({i}) $L_{\neg (bb)}$  состоит из всех слов, в которых нет двух
со\-сед\-них букв $b$.

({ii}) $L_{a?b}$ состоит из всех слов, содержащих подслово $a?b$
(напомню, что метасимвол $?$ обозначает любой символ из $\s$).
\end{Quest}

\begin{Answer}
Ниже будет рассмотрен алгоритм Кнута- Морисса- Пратта, который позволяет строить
такие языки весьма интеллигентно.
%Мы рассмотрим некоторые способы построения таких автоматов на следующих занятиях.
\end{Answer}

\begin{Definition}[Детерминированный автомат]
Автомат $A$ на\-зывается {\em детерминированным}, если для любых $q \in
Q, a \in \s$ множество $\delta(q,a)$ содержит не более одного
состояния. Если множество $\delta(q,a)$ всегда содержит точно одно
состояние, то $A$ называ\-ет\-ся {\em полным}. (В графе
переходов детерминированного автомата $A$ не может быть двух кратных
дуг, помеченных одинаковой буквой. У полного автомата дополнительно
возможен переход из любого состояния по любому входному символу.)
\end{Definition}

\begin{Rk}
Фактически конечный автомат иллюстрирует 
распространенный способ предикативного описания языков, когда вводится
некоторое абстрактное вычислительное устройство и
специфицируются допустимые вычисления. При этом сам про\-цесс вычисления
может и не иметь ``природных'' аналогов
Это от\-носится, например,  к
рассмотренному механизму {\bf индетерминизма}, который приходится
моделировать на детерминированных ус\-трой\-ствах.  Но вместе с тем, абстрактное
определение позволяет изучать структурные свойства.
\end{Rk}

}

\medskip

{\bf 4 ($4\times 0.02$ балла).} 
КА ${\cal A}$ задан диаграммой (финальное состояния обозначено черным кругом):

\unitlength=1mm
\begin{picture}(55,35)(-10,-10)
\multiput(0,-5)(20,0){2}{\circle{10}}
%\put(-6,6){{\small 0}}
\put(10,15){\circle*{2}}
\multiput(0,0)(20,0){2}{\circle{2}}
\put(0,0){\vector(2,3){9}}
\put(5,4){{\small 1}}
\put(-4,2){{\small $\mathbf{q_0}$}}
\put(10,15){\vector(2,-3){9}}
\put(14,4){{\small 0}}
\put(9,15){\vector(-2,-3){9}}
\put(4,9){{\small 1}}
\put(5,16){{\small $\mathbf{q_1}$}}
\put(-7,0){\vector(1,0){4}}
\put(9,15){\vector(-2,-3){9}}
%\put(6,9){{\small 1}}
\put(22,2){{\small $\mathbf{q_2}$}}
\put(13,-6){{\small 1}}
\put(-7,-6){{\small 0}}
\multiput(-3,-1)(20,0){2}{\vector(2,1){2}}
\put(21,0){\vector(-2,3){10}}
\put(16,9){{\small 0}}
\end{picture}
\medskip


({\em i}) Является ли ${\cal A}$ детерминированным?

({\em ii}) Опишите последовательность конфигураций ${\cal A}$ при обработке
слова $w=011001$. Верно ли, что $w\in L({\cal A})$?

({\em iii}) Приведите последовательность конфигураций автомата при обработке слова
$001100$.

({\em iv})  Укажите два слова,
принадлежащие языку $L(\A)$, и два слова, не принадлежащие языку
$L(\A)$
\smallskip





{\bf 5 ($0.05+0.1$ балла).} Какие и следующих языков в алфавите $\{0,1\}$ регулярны?

({\em i}) $L_3$ состоит из всех слов, имеющих больше чем $6$ символов,
в которых последний символ слова совпадает с символом,
стоящим на $7$-ой от конца слова позиции.
Например, $1010101\in L_3$, а $0000001\notin L_3$.

({\em ii}) $L_4$ состоит из всех слов, которые прочитанные как числа в двоичной записи
имеют остаток $1$ при делении на $3$.

Например, $001010$ $(111_2=7_{10}=2\times 3+1) \in L_4$, а $10001$
$(10001_2=17_{10}=5\times 3+2) \notin L_4$.



\medskip


{\footnotesize
\centerline{\bf ПОИСК ПОДСТРОК}

\smallskip

Для этого раздела полезно посмотреть книгу А.Шеня или книгу ``Алгоритмы'', гл. 34 в 1-м изд. или гл. 32 во 2-м изд.

 Шень А.
{\em Программирование: Теоремы и Задачи.} М.: МЦНМО, 1995.
(Электронный вариант: www.mccme.ru/free-books)

Кормен Т., Лейзерсон Ч., Ривест Р. {\em Алгоритмы.} 1-е изд.: М.: МЦНМО, 1999, 2-е изд.: ``Вильямс'', 200?.

\smallskip

Регулярные языки и задачи контекстного поиска
тесно связаны. Наша цель --- научиться строить эффективные ДКА
для поиска заданных подстрок в тексте.

Основную проблему
в задаче поиска подстроки в слове-образце составляют повторения, 
поскольку искомая подстрока может начинаться вовсе не там, где мы думаем.
Можно, конечно, тупо ``прикладывать'' подстроку ко всем ее возможным
положениям в слове-образце, но на это уйдет много времени и алгоритм
будет неэффективным (квадратичным). Один из более хитрых (и, следовательно, правильных)
подходов заключается в вычислении т.н.  {\em префиксной функции}
$l(\cdot): \ss \ra \ss$, что позволяет {\em регулярно} перечислять все
подозрительные места. (Излагаемая ниже процедура носит название {\em
алгоритма Кнута- Морисса- Пратта} (КМП).)

{\bf Определение.} {\em
Назовем префикс слова $x$ {\bf допустимым}, если он равен
суффиксу $x$.
Значение функции $l(x)$ на слове $x \in \ss$ равно} 
{\bf наибольшему собственному (т.е. не совпадающего со всем словом $x$)
допустимому префиксу $x$.}}
\smallskip

{\bf 5 ($0.02$ балла).} Найдите значение префикс-функции $l(\cdot)$ на
строке ``{\em abbabbbabbabbbabbbababbabba}''.

 

{\footnotesize
 Если вы используете полный перебор, то приведите его ПРОТОКОЛ (описание всех попыток).
Иначе приведите доказательство (например, какой нибудь сертификат 
или протокол работы КМП-алгоритма).

\medskip

\centerline{\bf КМП алгоритм}
\smallskip

Проверьте (или прочитайте в указанных источниках, например, в книге А.Шеня), что все собственные допустимые
префиксы слова $x$ содержатся в конечной (почему?) последовательности
итераций:  $\{l(x), l(l(x)),\dots\}$.
Это позволяет
вычислять значение префикс-функция на строке за {\em линейное} время.

Рассмотрим следующий инкрементальный алгоритм.
(Считаем, что строка записана в массиве $x[1],\dots,x[n]$, а {\em длины} слов $|l(x[1]x[2]\dots x[i])|\,
i=1,\dots,n$ записаны в массиве $L$.)

Инициализация: $i=1; L[1]=0; $

ПОКА $i <> n \{$

$len = L[i];$ 

(все более длинные слова неподходящие)

ПОКА $(x[len+1] <> x[i+1]) \& (len > 0) \; len=L[len];$

(начало не подходит, поэтому применяем к нему функцию $l(\cdot)$)

ЕСЛИ $x[len+1]=x[i+1] \{ L[i+1]=len+1 \}$

($x[1]\dots x[len]$--- самое длинное подходящее начало)

ИНАЧЕ   (подходящих нет)

 $\{L[i+1]=0\}$;

$i=i+1$

$\}$

Линейность алгорита следует из неравенств:
(число итераций на $i$-м шаге) $\leq L[i] - L[i+1] +1, \, i=1,2 \dots$.

Для построения линейного алгоритма поиска подстроки $y$ в слове $x$ достаточно добавить в алфавит
новый символ, например $\ast$, и вычислить  $l(y\ast x)$ по указанному способу.


Если строк несколько: $x_1, x_2, \dots,
x_k$, то можно, конечно, последовательно искать вхождение каждой
строки, но тогда алгоритм уже не является {\em линейным} по входу
(почему?).  Поэтому лучше (и правильнее) применить прежнюю идею
префикс-функции.  Для этого нужно ``склеить'' строки в корневое 
{\em дерево префиксов}\footnote{Кажется, эта процедура называется алгоритмом Ахо-Корасик.}. Ребра
дерева помечены буквами, а вершины --- допустимыми префиксами строк 
$x_1, x_2, \dots, x_k$. Метка корня равна $\eps$. А метка вершины $v$ (являющаяся префиксом одной из строк
$x_1, x_2, \dots, x_k$)
равна конкатенации меток ребер вдоль пути от корня до $v$. 
Например, если искать вхождение
строк из множества $aabca, aabba, aacba, aaca$, то получим дерево,
показанное на рисунке:}


\unitlength=1mm
\begin{picture}(105,45)(-15,-10)
\multiput(0,10)(10,0){6}{\circle*{2}}
\multiput(0,10)(10,0){5}{\vector(1,0){9}}
\put(5,11){{\small a}}
\put(15,11){{\small a}}
\put(25,11){{\small b}}
\put(35,11){{\small c}}
\put(45,11){{\small a}}
\multiput(40,0)(10,0){2}{\circle*{2}}
\put(30,10){\vector(1,-1){9}}
\put(32,4){{\small b}}
\put(40,0){\vector(1,0){9}}
\put(45,1){{\small a}}
\multiput(30,20)(10,0){3}{\circle*{2}}
\multiput(20,10)(10,10){2}{\vector(1,1){9}}
\put(22,14){{\small c}}
\multiput(30,20)(10,0){2}{\vector(1,0){9}}
\put(35,21){{\small b}}
\put(45,21){{\small a}}
\put(40,30){\circle*{2}}
\put(32,24){{\small a}}
\end{picture}

{\footnotesize ДКА, распознающий наличие одной из подстрок $x_1,
\dots, x_k$ во входном слове, двигается по вершинам дерева.
При этом {\em текущая} вершина --- это наибольшая (самая правая),
являющаяся концом прочитанной автоматом части слова $y$.
Иными словами, самый длинный суффикс прочитанной части,
являющийся началом одной из подстрок. Для быстрого вычисления
переходов автомата достаточно уметь вычислять префиксную 
функцию  $l(\cdot)$ на дереве Так же как и в случае обычной префикс-функции
просто показать, что если
$P$ --- вершина-префикс дерева, то
множество всех вершин дерева, являющихся концами $P$, равно $\{l(P),
l(l(P)), \dots\}$, и, таким образом, функцию переходов можно быстро
вычислять аналогично алгоритму КМП.}

\smallskip
{\bf 6 ($3\times 0.05$ баллов).} 
Постройте ПДКА КМП для распознавания языка 
$T\subseteq \{a,b\}^*$, определенного в задаче N $1$.
% (т.е. попробуйте
%``научно'' описать язык из задачи N $1$).
Можно, например,
рассмотреть дополнительный язык $\bar T$ (возможно, нужно специальным образом
учесть условие о том, что слова из $T$ начинаюnся с $b$ и заканчиваются $a$), 
построить для него ДКА, используя КМП-алгоритм.
А затем нужно придумать, как по полученным данным построить КА для искомого языка $T$.

\noindent ({\em i}) Постройте ПДКА КМП $\A_1$ для $\bar T$.


\noindent ({\em ii}) Постройте КА $\A_2$ для $T$

\noindent ({\em iii}) Приведите последовательность конфигураций  $\A_2$
для строки из задачи N $5$.

\smallskip

{\footnotesize Во второй части курса мы, возможно, рассмотрим еще два 
основанных на других идеях эффективных
алгоритма поиска подстрок, принадлежащие соответственно
 {\bf Boyer- Moore} и {\bf Рабину- Карпу}.
Подробности этих алгоритмов можно найти в указанных книгах.

}

\medskip 



\medskip
\centerline{\bf Индукция.}
\medskip


{\bf 7 ($0.3$ балла) Призовая задача\footnote{Призовые задачи можно решать в течение всего семестра. Их решение может принести серьезные бонусы при подведении итогов.}.} 
{\footnotesize Существует прямоугольный невыпуклый шестиугольник,
который можно разбить на два подобных с коэффициентами подобия 
$\alpha$ и $\alpha^2$ для некоторого $\alpha <1$ (см., рис. 1).

\unitlength=1mm
\begin{figure}



\begin{picture}(90,100)(-10,0)
\put(31,-3){{\small $\alpha$}}
\put(0,0){\line(1,0){64}}
\put(-3,41){{\small $1$}}
\put(0,0){\line(0,1){84}}
\put(19,85){{\small $\alpha^3$}}
\put(0,84){\line(1,0){39}}
\put(40,67){{\small $\alpha^4$}}
\put(39,84){\line(0,-1){32}}
\put(31,53){{\small $\alpha^7$}}
\put(51,53){{\small $\alpha^5$}}
\put(25,52){\line(1,0){39}}
\put(21,40){{\small $\alpha^6$}}
\put(25,32){\line(0,1){20}}
\put(12,29){{\small $\alpha^5$}}
\put(1,15){{\small $\alpha^4$}}
\put(1,57){{\small $\alpha^5$}}
\put(0,32){\line(1,0){25}}
\put(65,25){{\small $\alpha^2$}}
\put(64,0){\line(0,1){52}}
\end{picture}

\caption{Разбиение шестиугольника.}

\end{figure}

\medskip

Сравнивая площади, получаем, что $\alpha^4+\alpha^2=1$.

Разбиения можно итеративно повторять, причем удобно считать, что в
каждом разбиении участвуют только шестиугольники двух видов: б\'ольшие и
меньшие, и на каждой итерации меньшие шестиугольники объявляются
большими и не разбиваются, а каждый б\'ольший --- разбивается на 
б\'ольший и меньший. Назовем {\em канонической}  окраску сторон
 шестиугольников, изображенную на рисунке 2, а также будем считать,
что окрашенные части ориентированы.

\unitlength=1mm
\begin{figure}



\begin{picture}(90,100)(-10,0)
\put(6,1){{\small $7$}}
\put(0,0){\vector(1,0){14}}
\put(26,1){{\small $5$}}
\put(14,0){\vector(1,0){25}}
\put(51,1){{\small $5$}}
\put(64,0){\vector(-1,0){25}}
\put(1,73){{\small $6$}}
\put(0,84){\vector(0,-1){20}}
\put(1,47){{\small $4$}}
\put(0,64){\vector(0,-1){32}}
\put(1,15){{\small $4$}}
\put(0,0){\vector(0,1){32}}
\put(31,81){{\small $7$}}
\put(39,84){\vector(-1,0){14}}
\put(12,81){{\small $5$}}
\put(25,84){\vector(-1,0){25}}
\put(36,67){{\small $4$}}
\put(39,84){\vector(0,-1){32}}
\put(31,53){{\small $7$}}
\put(31,49){{\small $7$}}
\put(51,49){{\small $5$}}
\put(25,52){\vector(1,0){14}}
\put(39,52){\vector(1,0){25}}
\put(22,40){{\small $6$}}
\put(26,40){{\small $6$}}
\put(25,32){\vector(0,1){20}}
\put(12,29){{\small $5$}}
\put(12,33){{\small $5$}}
\put(0,32){\line(1,0){25}}
\put(61,15){{\small $4$}}
\put(64,52){\vector(0,-1){20}}
\put(61,41){{\small $6$}}
\put(64,32){\vector(0,-1){32}}
\end{picture}

\caption{Окраска. Цвета --- цифры.}

\end{figure}

\medskip
 
Назовем разбиение {\em правильным}, если каноническая раскраска
входящих в него б\'ольших и меньших шестиугольников согласована у всех
соседей по цвету и ориентации, т.е. каждая стрелка
примыкает к стрелке того же цвета и направления.
}

Докажите по  индукции, что на любой итерации получается 
правильное разбиение.





\smallskip

{\em  В заключение напоминаю, что при построении любого объекта с
нужным свойством необходимо {\bf ДОКАЗЫВАТЬ}, что объект этим свойством
обладает}.

\smallskip

\centerline{\bf ДОПОЛНИТЕЛЬНЫЕ ЗАДАЧИ}
{\footnotesize
\centerline{\bf Эти задачи НЕ ОЦЕНИВАЮТСЯ,}
\centerline{\bf но будут проверены,}
\centerline{\bf если вы пришлете решения.}
\centerline{\bf Баллы приведены для информации} 
\centerline{\bf о сложности задачи!}

\medskip

\centerline{\bf Индукция }

\smallskip

{\bf 1 ($2\times 0.05$ балла).} Дано клетчатое поле размером $2^n\times 2^n\, n>1000$ и фишка в форме буквы $L$,
покрывающая $3$ клетки. Можно ли покрыть поле $L$-фишками: 

({\em i}) если удалить $2$ крайние угловые клетки поля: северо-западную и юго-восточную;

({\em ii}) если удалить крайнюю северо-западную  клетку поля.

\smallskip


{\bf 2 ($0.15+0.15$ балла).} ({\em i}) 
Верно ли следующее индуктивное утверждение? (Вы должны указать
{\em конкретные места ошибок}, если они
есть.)

Для рекурентности
\begin{displaymath}
T(n)=\left\{
\begin{array}{c}
20, \; \mbox{ если } \; n \leq 20,\\
T(n - \lfloor \sqrt{n} \rfloor) + T(\lceil \sqrt{n} \rceil) + O(n), \;
\mbox{ иначе.}
\end{array}
\right.
\end{displaymath}
справедлива оценка: $T(n) = O(n\log n)$
($\lfloor a \rfloor$ и $\lceil a \rceil$ обозначают, соответственно,
максимальное целое число не большее, чем $a$ и минимальное целое не
меньшее, чем $a$).

%\footnote{
%$f(n)=\Omega(g(n)) \;
%\Leftrightarrow \;  g(n) = O(f(n));
%$f(n)=\Theta(g(n)) \;
%\Leftrightarrow \; const_1 g(n) \leq f(n) \leq const_2 g(n)$ для всех
%$n \geq n_0$}.

Предположим, что асимптотическая формула верная. Достаточно проверить
индуктивный переход.  Имеем: $(n - \sqrt{n})\log(n-\sqrt{n}) =
(n - \sqrt{n})(\log n + \log (1 - \frac{1}{\sqrt{n}})) = n \log n -
\sqrt{n} \log n + (n - \sqrt{n})\log (1- \frac{1}{\sqrt{n}})$.
Отсюда получаем:
%\sgrt{n} \log n + (n - \sqrt{n}) \log(1 - \frac{1}{\sqrt{n}}). Поэтому:
$n\log n = (n - \sqrt{n}) \log (n -
\sqrt{n}) + \sqrt{n} \log \sqrt{n} +  [\sqrt{n} \log \sqrt{n} -
(n - \sqrt{n})\log (1- \frac{1}{\sqrt{n}})]$. Но очевидно, что
выражение в квадратных скобках есть $O(n)$. Таким образом,
утверждение доказано.

\noindent ({\em ii}) Оцените истинный порядок роста рекуррентности $T(n)$.
(Конечно, если приведенное выше рассуждение верно, то $T(n) = O(n\log n)$.)


\medskip

\centerline{\bf Слова, языки, операции над языками} 
\centerline{\bf (Гомо)морфизмы}

\smallskip



{\bf 3 ($0.4$ балла).} 
{\footnotesize Для решения этой задачи нужно иметь МИНИМАЛЬНОЕ представление 
об игре в шахматы: нужно понимать, что игра определяется
последовательностью ходов и знать, как ходят отдельные фигуры и как игра заканчивается. 

В частности, согласно т.н. {\em немецкому правилу}, игра 
заканчивается (ничьей), если 
некоторая серия ходов {\em дважды последовательно 
повторяется и затем следует первый ход серии.}
}

Существует ли бесконечная шахматная партия, если для остановки игры
используется ``немецкое правило''?
\smallskip





{\bf 4 ($0.3+0.2$ балла).} 
Пусть $X=\lim_{n\to \infty} X_n$, где двоичные слова $X_i, i=0,1,\dots$
определяются индуктивно $X_0=0, X_1=01, 
X_3=0110,\dots, X_{n+1}=X_n\overline{X_n},\dots$ (замена 
в двоичном слове $x$ всех нулей на 
единицы и наоборот обозначается $\bar x$).

$Y=y_0y_1\dots$ --- это двоичное слово, в котором 
$n$-й символ $y_n$ равен сумме по $\pmod{2}$ разрядов двоичного 
разложения числа $n$, например, $45=101101_2$ и $y_{45}=1+0+1+1+0+1 
\pmod{2}=0$.

\noindent ({\em i})  Докажите, что $X=Y$.

Определим на множестве двоичных слов морфизм:
$\mu(0)=01, \mu(1)=10$. Получаем: $\mu(0)=01, \mu^2(0)=\mu\mu(0)=
0110,\dots$ 

\noindent ({\em ii})  Докажите, что $\mu^n(0)=X_n$.

\smallskip

{\bf 5 ($0.1+0.8$ балла).} Рассмотрим последовательность : $\frac12,
 \frac{\frac12}{\frac34}, 
\frac{\frac{\frac12}{\frac34}}{\frac{\frac56}{\frac78}},\dots$
Нас будет интересовать ее предел $\lambda$.

\noindent ({\em i}) Докажите, что $\lambda=\lim_{n\to \infty} 
\prod_{n\geq 0}{\frac{2n+1}{2n+2}^{(-1)^{t_n}}}$.

\noindent ({\em ii}) Докажите, что $\lambda=\frac1{\sqrt2}$.






\smallskip

{\bf 6 ($3\times 0.03+0.25+1$ балл).} 
{\footnotesize Каждое натуральное число можно однозначно представить в виде
$n=2^u(4s+t)$, здесь $u,s,t$ --- натуральные числа; 
$2^u$ --- это наибольшая степень двойки, на которую делится
$n$ и $t<4$. По построению $t$ либо $1$, либо $3$.
Обозначим $\sigma$ бесконечную 
последовательность значений $t$, когда $n$ последовательно 
пробегает все натуральные числа. Запишем $\sigma$ в виде
бесконечного слова в алфавите $\{1,3\}$.\footnote{Замечу, что если
специальным образом интерпретировать $1$ и $3$ в слове $\sigma$, как повороты направо и 
налево, то можно рисовать интересные картинки. 
%Возможно мы
%поговорим об этом на семинаре.
}

}

\noindent ({\em i})  Вычислите слово $\sigma_{15}$,
состоящего из первых $15$ символов $\sigma$.

\noindent ({\em ii})  Покажите, что символы, стоящие на нечетных позициях
$\sigma$ чередуются.

\noindent ({\em iii}) Определим индуктивно последовательность слов 
$\{x_i,y_i\},\, i=1,2,\dots$
в алфавите $\{1,3\}$: $x_1=1,\, y_1=3,\, x_{i+1}=x_i 1 y_i,\,
y_{i+1}=x_i 3 y_i,\, i=1,2,\dots$. Проверьте, что $x_4=\sigma_{15}$.

\noindent ({\em iv}) Покажите, что при всех $i$ $x_i$ является префиксом
$\sigma$, что можно записать в виде $\sigma=\lim_{i\to \infty} x_i$.


\noindent ({\em v}) {\bf Призовая задача} Докажите или опровергните,
что $\sigma$ является неподвижной точкой морфизма.

}



\end{document}