\documentclass[10pt]{article}

\usepackage[utf8]{inputenc}
\usepackage[T2A]{fontenc}
\usepackage[russian, english]{babel}

\usepackage{amssymb, amsmath, textcomp, tabularx, graphicx}

\newcolumntype{C}{>{\centering\arraybackslash}X}%

\title{Задание 3}
\author{Коновалов Андрей, 074}
\date{}

\let \eps \varepsilon

\begin{document}

\maketitle

\noindent
\begin{tabularx}{\textwidth}{|C|C|C|C|C|C|C|}
  \hline
  0 & 1 & 2 & 3 & 4 & 5 & $\sigma$ \\
  \hline
  &&&&&& \\
  \hline
\end{tabularx}

\bigskip

{\bf Задача 0}

{\bf (ii).} Для начала научимся из множества состояний $Q$ автомата ${\cal A}$ строить множество состояний $Q_{\eps}$, каждый элемент которого достижим по $\eps$-переходам из какого-то из состояний $Q$. Конкретно в данном случае мы будем добавлять элементы в $Q$ в некотором порядке и в итоге получим множество $Q_{\eps}$.

При построении будем использовать конструкцию "очередь". Это некоторый объект, над которым можно совершать 2 операции: положить туда элемент и достать оттуда элемент. Причем элементы достаются в том же порядке, в котором кладутся.

Для построения множества $Q_{\eps}$ будем использовать алгоритм, известный как поиск в ширину. Сопоставим каждому состоянию ${\cal A}$ элемент множества $\{ true, false \}$. Состояния, помеченные $true$ будем называть посещенными, а Состояния, помеченные $false$ - не посещенными. Изначально пометим все состояния ${\cal A}$ как непосещенные.

Пусть дано множество $Q$. Пометим все состояния из $Q$ как посещенные. Возьмем пустую очередь и последовательно положим туда все элементы из $Q$. Теперь, до тех пор пока очередь не пуста, будем доставать из нее элементы. Пусть мы достали из очереди элемент $q$. Просматривая все переходы из состояния $q$ рассматриваем лишь $\eps$-переходы. Пусть существует $\eps$-переход из $q$ в состояние $p$. Если $p$ помечено как посещенное, то пропускаем его, иначе - помечаем как посещенное, добавляем его в множество $Q$ и кладем в очередь. Если очередь пуста, то мы построили множество $Q_{\eps}$.

Понятно, что таким образом в множестве $Q_{\eps}$ окажутся все вершины, достижимые по $\eps$-переходам из какого-либо из состояний $Q$ и никакие другие.

Оценим время работы алгоритма. Всего через очередь пройдет $O(|{\cal A}|)$ вершин. Для каждой из них мы просмотрим все переходы из нее, что для какой-то одной вершины занимает время $O(|{\cal A}|)$. Итого, общее время работы: $O({|{\cal A}|}^2)$, где $|{\cal A}|$ - количество вершин в автомате ${\cal A}$.

\smallskip

Пусть мы построили автома ${\cal A}$ по заданному регулярному выражению $R$ за $O(|R|)$, причем $|{\cal A}| = O(|R|)$. Проверим теперь, принимает ли ${\cal A}$ слово $w$. Для этого будем индуктивно вычислять множества $V_i$ состояний, достижимых из начальных после прочтения префикса длины $i$ слова $w$.

Посмотрим на множество из всех начальных состояний. Дополним его до множества состояний, достижимых из начальных по $\eps$-переходам, используя описанный выше алгоритм и назовем полученное множество $V_0$. Далее, пусть построено множество $V_i, i < |w|$. Построим $V_{i + 1}$. Пусть $V_{i + 1} = \varnothing$. Пусть $(i + 1)$-ая буква слова $w$ есть $a$. Переберем все элементы множества $V_i$ и для каждого из них просмотрим все переходы, помеченные буквой $a$. Состояния, в которые ведут эти переходы добавим в $V_{i + 1}$. Далее дополним $V_{i + 1}$ состояниями, достижимыми по $\eps$-переходам, используя алгоритм, описанный выше. Понятно, что в множестве $V_{i + 1}$ окажутся лишь те состояния, которые достижимы из какого-либо состояния множества $V_i$ после обработки буквы $a$, а значит там окажутся только те состояния, которые достижимы из какого-либо из начальных состояний после обработки префикса длины $i + 1$ слова $w$.

Получаем, что слово $w$ принимается тогда и только тогда, когда $V_{|w|}$ содержит хотя бы одно финальное состояние.

Оценим время работы итогового алгоритма. Для построения множества $V_i$ необходимо перебрать все переходы, которых $O(|{\cal A}|)$, для каждого из элементов $V_{i - 1}$, которых $O(|{\cal A}|)$. Для каждого из множеств $V_i$ мы используем поиск в ширину, сложность которого $O({|{\cal A}|}^2)$. Всего множеств $O(|w|)$. Учитывая время, затраченное на построение ${\cal A}$ получаем $O(|R|^3 |w|)$.

\medskip

{\bf Задача 1}

{\bf (i).}
Пусть задана диаграмма автомата ${\cal A}$. Построим по ней леволинейную грамматику $G$. Аксиомой объявляется новый нетерминал $S$. Начальное состояние $q_0$ порождает вывода $Q_0 \rightarrow \eps$. Каждое финальное состояние $q_i$ порождает правило $S \rightarrow Q_i$. Каждая дуга $(q_{init} \rightarrow q_{end})$, помеченная символом $a$, порождает правило $Q_{end} \rightarrow Q_{init} a$.

\smallskip

{\bf (ii).}
Пусть задан автомат ${\cal A} = (Q = \{ q_0, q_1, \dots \}, \Sigma, \delta, q_0, F)$. Построим по нему леволинейную грамматику $G = ({\cal Q} = \{ S, Q_1, Q_2, \dots \}, \Sigma, P, S)$, где $P = \{ S \rightarrow Q_i$ для всех $q_i \in F \} \cup \{ Q_i \rightarrow Q_j a$, если $\delta (q_j, a) = q_i \} \cup \{ Q_0 \rightarrow \eps \}$.

\smallskip

{\bf Доказательство.}
Поскольку оба представления автомата эквиваленты, при доказательстве будем пользоваться представлением ввиде диаграммы. Заметим, что правило вывода $Q_i \rightarrow Q_j a$ входит в $G$ тогда и только тогда, когда в ${\cal A}$ существует ребро из состояния $q_j$ в состояние $q_i$, имеющее метку $a$.

1. Докажем $L({\cal A}) \subseteq L(G)$. Возьмем произвольное слово $w \in L({\cal A})$. Поскольку $w \in L({\cal A})$, то существует некоторая последовательность состояний автомата $Z = (q_0, \dots, q_{n - 1}, q_n)$, причем $q_0$ - начальное состояние, $q_n$ - одно из финальных состояний. Так же для любой пары последовательных состояний $(q_i, q_{i + 1})$ из $Z$ в ${\cal A}$ существует ребро $q_i \rightarrow q_{i + 1}$ с некоторой пометкой $m$, причем последовательная конкатенация пометок для каждой пары является словом $w$.

Построим, используя эту последовательность состояний, вывод слова $w$ в грамматике $G$. Поскольку $q_n$ - финальное состояние, то существует правило вывода $S \rightarrow Q_n$. Применим это правило. Поскольку для любой пары последовательных состояний $(q_i, q_{i + 1})$ из $Z$ в ${\cal A}$ существует ребро $q_i \rightarrow q_{i + 1}$ с некоторой пометкой $m$, то в $G$ существует правило вывода $Q_{i + 1} \rightarrow Q_i m$. Применим последовательно правила вывода в следующем порядке: $Q_n \rightarrow Q_{n - 1} m_{n - 1}$, $Q_{n - 1} \rightarrow Q_{n - 2} m_{n - 2}$, $\dots$, $Q_1 \rightarrow Q_0 m_0$. Видно, что после этого будет выведено слово вида: $Q_0 w$. Поскольку $q_0$ - начальное состояние, то существует правило вывода $Q_0 \rightarrow \eps$. После применения этого последнего правило будет выведено слово $w$, а значит $w \in L(G)$.

2. Докажем $L(G) \subseteq L({\cal A})$. Доказательство аналогично изложенному выше. Возьмем произвольное слово $w \in L(D)$. Для него существует вывод в $G$, состоящий из последовательного применения правил: $S \rightarrow Q_n, Q_n \rightarrow Q_{n - 1} m_{n - 1}$, $\dots$, $Q_1 \rightarrow Q_0 m_0, Q_0 \rightarrow \eps$. Из существования такого вывода следует существования последовательно состояний $(q_0, \dots, q_n)$, причем при обработке слова $w$ автомат ${\cal A}$ пройдет через эту последовательность состояний начиная из начального состояний $q_0$, достигнет финальной вершины $q_n$ и примет слово $w$.

3. $L({\cal A}) \subseteq L(G) \wedge L(G) \subseteq L({\cal A}) \Rightarrow L(G) = L({\cal A})$.

\medskip

{\bf Задача 2}

{\bf (i).}
Построим леволинейную грамматику $G = (N, T, P, S)$ используя алгоритм, описанный в задаче 1. Множеством нетерминальных символов будет являться множество $N = \{ S, Q_0, Q_1, Q_2 \}$. Множеством терминальных символов будет являться множество $T = \{ 0, 1 \}$. Множество выводов $P$ получаем, рассмотрев все ребра диаграммы.
\begin{align*}
  S &\rightarrow Q_1 \\
  Q_0 &\rightarrow Q_0 0 \\
  Q_0 &\rightarrow Q_1 1 \\
  Q_0 &\rightarrow \eps \\
  Q_1 &\rightarrow Q_0 1 \\
  Q_1 &\rightarrow Q_2 0 \\
  Q_2 &\rightarrow Q_2 1 \\
  Q_2 &\rightarrow Q_1 0
\end{align*}

То, что грамматика - искомая, следует из корректности построения грамматики по автомату, доказанной в задаче 1.

\smallskip

{\bf (ii)}.
Используя описанную в теории последовательность действий построим определяющую систему $D(G)$ для грамматики $G$.

$$
\begin{cases}
  Q_0 = Q_0 0 + Q_1 1 + \eps \\
  Q_1 = Q_0 1 + Q_2 0 \\
  Q_2 = Q_1 0 + Q_2 1
\end{cases}
$$

Для нахождения наименьшей неподвижной точки этой системы потребуется заметить, что наименьшей неподвижной точкой уравнений $X = X \alpha + \beta$ будет $X = \beta \alpha^*$. В этом можно убедится совершенно аналогичным описанному в теории способом. Найдем эту точку.
\begin{align*}
  Q_2 &= Q_2 1 + Q_1 0 \\
  Q_2 &= Q_1 0 (1)^* = Q_1 (0 (1)^*) \\
  Q_1 &= Q_0 1 + Q_1 (0 (1)^* 0) \\
  Q_1 &= Q_0 (1 (0 (1)^* 0)^*) \\
  Q_0 &= Q_0 0 + Q_0 (1 (0 (1)^* 0)^* 1) + \eps \\
  Q_0 &= \eps ((1(0 (1)^* 0)^* 1) + 0)^* = (1(0 (1)^* 0)^* 1 + 0)^* \\
  Q_1 &= (1(0 (1)^* 0)^* 1 + 0)^* (1 (0 (1)^* 0)^* ) \\
  Q_2 &= (1(0 (1)^* 0)^* 1 + 0)^* (1 (0 (1)^* 0)^* ) (0 (1)^*)
\end{align*}

В последних трех строчках вывода написано решение системы, которое является ее наименьшей неподвижной точкой.

\medskip

{\bf Задача 3}

Проверим, что оба языка $L_1$ и $L_2$ удовлетворяют сильному варианту ЛР:
$$\exists C \; \forall w \in L_1: |w| \geq C, \; \exists x, y, z: w = xyz, |xy| \leq C, |y| \geq 0 \, \rightarrow \, \forall i \geq 0 \; x y^i z \in L_1$$

{\bf (i).}
Язык $L_1 = \{ a^n b^{2m}, m, n \geq 0 \}$. Возьмем $C = 2$. Рассмотрим произвольное слова $w \in L_1$. Это слово либо содержит ненулевое количество букв $a$, либо не содержит ни одной. Разберем эти 2 случая.

1. $w = a^n b^{2m} = xyz$. Слово $w$ содержит ненулевое количество букв $a$ $\Rightarrow n \geq 1$. Возьмем $x = \eps$, $y = a$, $z = a^{n - 1} b^{2m}$. Видно, что $|xy| = 1 \leq C = 2$, $|y| = 1 > 0$. Заметим, что $\forall i \geq 0 \; x y^{i} z = \eps a^{i} a^{n - 1} b^{2m} = a^{n - 1 + i} b^{2m} \in L_1$. Поскольку $n \geq 1$, то $n - 1 + i \geq 0$, а значит $\forall i \geq 0 \; x y^{i} z = a^{n - 1 + i} b^{2m} \in L_1$.

2. $w = b^{2m} = xyz$. $|w| \geq C = 2 \Rightarrow m \geq 1$. Возьмем: $x = \eps$, $y = b^2$, $z = b^{2m - 2}$. Видно, что $|xy| = 2 \leq C = 2$, $|y| = 2 > 0$. Заметим, что $\forall i \geq 0 \; x y^{i} z = \eps b^{2i} b^{2m - 2} = b^{2m - 2 + 2i}$. Поскольку $m \geq 1$, то $2m - 2 + 2i \geq 0$, а значит $\forall i \geq 0 \; x y^{i} z = b^{2m - 2 + 2i} \in L_1$.

Получаем, что в обоих случаях ЛР выполняется.

\smallskip

{\bf (ii).}
Язык $L_2 = \{ w \in \{ a, b \}^* | a^k b^{2^n}, k, n \geq 0 \} \cup \{ b^i, i \geq 0 \} = T_1 \cup T_2$. Возьмем $C = 1$. Рассмотрим произвольное слово $w \in L_2$. Это слово либо содержит ненулевое количество букв $a$, либо не содержит ни одной. Разберем эти 2 случая. 

1. Поскольку $w$ содержит $a$, то $w \in T_1$, а значит $w = a^k b^{2^n} = xyz$, причем $k \geq 1$. Возьмем: $x = \eps$, $y = a$, $z = a^{k - 1} b^{2^n}$. Видно, что $|xy| = 1 \leq C = 1$, $|y| = 1 > 0$. Заметим, что $\forall i \geq 0 \; x y^{i} z = \eps a^i a^{k - 1} b^{2^n} = a^{k - 1 + i} b^{2^n}$. Поскольку $k \geq 1$, то $k - 1 + i \geq 0$, а значит $\forall i \geq 0 \; x y^{i} z = a^{k - 1 + i} b^{2^n} \in T_1 \subseteq L_2$.

2. $w = b^n = xyz$. $|w| \geq C = 1 \Rightarrow n \geq 1$. Возьмем: $x = \eps$, $y = b$, $z = b^{n - 1}$. Видно, что $|xy| = 1 \leq C = 1$, $|y| = 1 > 0$. Заметим, что $\forall i \geq 0 \; x y^{i} z = \eps b^i b^{n - 1} = b^{n - 1 + i}$. Поскольку $n \geq 1$, то $n - 1 + i \geq 0$, а значит $\forall i \geq 0 \; x y^{i} z = b^{n - 1 + i} \in T_2 \subseteq L_2$.

Получаем, что в обоих случаях ЛР выполняется.

\medskip

{\bf Задача 4}

Заметим, что $L_2 = (a^* b^* c^*)^* = \{ a, b, c \}^* = \{ a, b, c \}^* \cdot \{ \eps \}$, а значит $L_2$ - простой.

Докажем, что язык $L_1 = (a^* b^* c)^*$ - не простой. Допустим, что он простой, то есть $\exists$ конечные $S_1, S_2: \; L_1 = S_2^* S_1$. Множество слов $a^* c \subset L_1$, а значит $a^* c \subset S_2^* S_1$. Поскольку множество слов $a^* c$ бесконечное, а множества $S_1$ и $S_2$ конечные, то $\exists \, n \geq 1: \; a^n \in S_2$. Аналогично бесконечное множество слов $b^*c \subset L_1$, а значит $\exists \, m \geq 1: \; b^m \in S_2$. Получаем, что существует слово $w$ с префиксом $b^m a^n$, $w \in S_2^* S_1 = L_1$. Слово $w$ имеет подслово $ba$, но $L_1 = (a^* b^* c)^*$ не может содержать слово с подсловом $ba$, поскольку между стоящей где-то буквой $b$ и стоящий после нее буквой $a$ должна стоять буква $c$. Получаем противоречие, значит $L_1$ - не простой.

Докажем, что язык $L_3 = (a^* + b^* c)^*$ - не простой. Допустим, что он простой, то есть $\exists$ конечные $S_1, S_2: \; L_3 = S_2^* S_1$. Множество слов $a^* \subset L_3$, а значит $a^* \subset S_2^* S_1$. Поскольку множество слов $a^*$ бесконечное, а множества $S_1$ и $S_2$ конечные, то $\exists \, n \geq 1: \; a^n \in S_2$. Аналогично бесконечное множество слов $b^*c \subset L_1$, а значит $\exists \, m \geq 1: \; b^m \in S_2$. Получаем, что существует слово $w$ с префиксом $b^m a^n$, $w \in S_2^* S_1 = L_3$. Слово $w$ имеет подслово $ba$, но $L_3 = (a^* + b^* c)^*$ не может содержать слово с подсловом $ba$, поскольку между стоящей где-то буквой $b$ и стоящий после нее буквой $a$ должна стоять буква $c$. Получаем противоречие, значит $L_3$ - не простой.

\end{document}
