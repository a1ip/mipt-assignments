\documentclass[10pt]{article}

\usepackage[cp1251]{inputenc}
\usepackage[T2A]{fontenc}
\usepackage[russian, english]{babel}

\usepackage{amssymb, amsmath, textcomp, tabularx, graphicx}

\newcolumntype{C}{>{\centering\arraybackslash}X}%

\title{Задание 4}
\author{Коновалов Андрей, 074}
\date{}

\let \eps \varepsilon

\begin{document}

\maketitle

\noindent
\begin{tabularx}{\textwidth}{|C|C|C|C|C|C|C|}
  \hline
  1 & 2 & 3 & 4 & 5 & 6 & $\sigma$ \\
  \hline
  &&&&&& \\
  \hline
\end{tabularx}

\bigskip

{\bf Задача 1}

{\bf (ii).}
Контрпример: язык $L$ в алфавите $\{ a, b \}$ задается регулярным выражением: $a \cap b$, морфизм $h$ определяется: $h(a) = a$, $h(b) = a$.

Видно, что язык $L = \varnothing$, а значит $h(L) = \varnothing$. Подставим в регулярное выражение значения морфизма $h$ на литералах, получим: $h(a) \cap h(b) = a \cap a = a$. Но $\{ a \} \neq \varnothing$.

\medskip

{\bf Задача 2}

{\bf (i).}
Пересечем данный язык $L_1$ с регулярным языком, заданным регулярным выражением $0(1)^*$. Получим язык $L = \{ 0 1^{2^i} | i \geq 0 \}$.

Докажем, что язык $L$ - нерегулярный. Допустим, что он регулярный, тогда он удовлетворяет лемме о разрастании с некоторой константой $p$. Возьмем слово $w = 0 1^{2^p}$. Его длина больше, чем $p$, следовательно существует такое разбиение $w = xyz$, что $0 < |y| \leq C$ и для всех $i \geq 0$, $x y^i z \in L$. Возможны только следующие случаи:

1. $x = \eps$. В этом случае $y$ имеет префикс $0$, поскольку $|y| > 0$. Но тогда при $i = 2$ слово $x y^i z$ будет содержать две буквы $0$, следовательно $\notin L$.

2. $x \neq \eps$. В этом случае $y = 1^n$ для некоторого $n \geq 1$. Заметим, что количество слов вида $x y^i z$, длиной не больше, чем некотрое число $C$ растет, в зависимости от $C$, линейно. При этом число слов в языке $L$, длиной не более $C$ растет логарифмически. Следовательно, язык $L$ не может содержать все слова вида $x y^i z$.

В обоих случаях мы пришли к противоречию, следовательно $L$ - нерегулярный, а значит и язык $L_1$ - нерегулярный.

\smallskip

{\bf (ii).} Пересечем данный язык $L_2$ с регулярным языком, заданным регулярным выражением $0 (1)^* 0 (1)^*$. Поссмотрим на результирующий язык $L$. Поскольку любое его слово $w$ должно быть квадратом некоторого слова $v$, и $w$ должно содержать ровно две буквы $0$ можно сделать вывод о том, что $v$ содержит ровно одну букву $0$. Поскольку первая половина $w$ должна начинаться с буквы $0$, то и вторая тоже. А значит $w$ имеет вид $0(1)^n 0 (1)^n$. Получаем $L = \{ 0 (1)^n 0 (1)^n | n \geq 0 \}$.

Докажем, что язык $L$ - нерегулярный. Допустим, что он регулярный, тогда он удовлетворяет лемме о разрастании с некоторой константой $p$. Возьмем слово $w = 0 1^p 0 1^p$. Его длина больше, чем $p$, следовательно существует такое разбиение $w = xyz$, что $0 < |y| \leq C$ и для всех $i \geq 0$, $x y^i z \in L$. Возможны только следующие случаи:

1. $y$ содержит букву $0$. В этом случае при $i = 2$ поличество букв $0$ в слове $x y^i z$ будет больше двух, а значит $x y^2 z \notin L$.

2. $y = 1^n$ для некоторого $n \geq 1$. Но в этом случае для $i = 2$ количество букв $1$ до второго вхожения буквы $0$ будет отличаться от количества букв $1$ после этого вхождения на $n$, а значит $x y^2 z \notin L$.

В обоих случаях мы пришли к противоречию, следовательно $L$ - нерегулярный, а значит и язык $L_2$ - нерегулярный.

\medskip

{\bf Задача 3}

Нет, поскольку при $n = m = 0 \rightarrow nm = 0$, а значит $(0, 0) \notin R$, следовательно $R$ - не рефлексивное, следовательно $R$ - не отношение эквивалентности.

\medskip

{\bf Задача 5}

{\bf (i)}
Построим ПДКА $A$ для $L$. Он изображен на диаграмме ниже.

\centerline{\includegraphics{{5.1}.png}}

Докажем его корректность по индукции по длине $n$ слова $w$.

{\it База.} При $n = 0$ автомат принимает слово $\eps$. При $n = 1$ автомат принимает слова $0$ и $1$. Свойство выполняется. База доказана.

{\it Переход.} Пусть автомат принимает только те слова длины меньше $n$, которые не содержат двух букв $1$ подряд. Докажем, что аналогичное утверждение выполняется для слов длины $n$.

Возьмем слово $w$ длины $n$. Пропустим через $A$ префикс $v$ длины $n - 1$ слова $w$. Возможны два случая:

1. $A$ не принимает $v$. Это означает, что $v$ содержит две буквы $1$ подряд, а значит и $w$ их содержит и не должно быть принято. Заметим, что $A$ после обработки $v$ окажется в единственном нефинальном состоянии $q_3$. Какой бы ни была последняя буква слова $w$, $A$ так и останется в $q_2$, а значит $w$ не будет принято. Необходимое свойство выполняется.

2. $A$ принимает $v$, а значит находится либо в состоянии $q_0$ или в состояниее $q_2$. Если последняя буква $v$ была $1$, а последняя буква $w$ тоже $1$, то $A$ перейдет в $q_3$ и $w$ не будет принято. Если последняя буква $v$ была $1$, а последней буквой $w$ является $0$, то $A$ перейдет в $q_0$ и $w$ будет принято. Если последняя буква слова $v$ не существовала или была $0$, то $A$ после обработки $v$ окажется в $q_0$, и при переходе по последней букве слова $w$ перейдет или в $q_0$ или в $q_1$ и $w$ будет принятно. Необходимое свойство выполняется.

Переход доказан.

\smallskip

{\bf (ii)}
Построим праволинейную грамматику $G = (\{ Q_0, Q_1, Q_2 \}, \{ 0, 1, \eps \}, P, Q_0)$ по автомату $A$. Множество выводов $P$ будет выглядеть следующим образом:

\begin{align*}
  Q_0 &\rightarrow 0 Q_0 \\
  Q_0 &\rightarrow 1 Q_1 \\
  Q_1 &\rightarrow 0 Q_0 \\
  Q_1 &\rightarrow 1 Q_2 \\
  Q_2 &\rightarrow 0 Q_2 \\
  Q_2 &\rightarrow 1 Q_2 \\
  Q_0 &\rightarrow \eps \\
  Q_1 &\rightarrow \eps
\end{align*}

Составим систему регулярных уравнений по полученной грамматике.

$$
\begin{cases}
  Q_0 = 0 Q_0 + 1 Q_1 + \eps \\
  Q_1 = 0 Q_0 + 1 Q_2 + \eps \\
  Q_2 = 0 Q_2 + 1 Q_2
\end{cases}
$$

Найдем наименьшую неподвижную точку этой системы.
\begin{align*}
  Q_2 &= (0 + 1) Q_2 \\
  \eps \notin \{ 0, 1 \} \Rightarrow Q_2 &= \varnothing \\
  Q_1 &= 0 Q_0 + \eps \\
  Q_0 &= 0 Q_0 + 1 Q_1 + \eps \\
  Q_0 &= 0 Q_0 + 1 (0 Q_0 + \eps) + \eps \\
  Q_0 &= (0 + 10) Q_0 + (1 + \eps) \\
  Q_0 &= (0 + 10)^* (1 + \eps) \\
  Q_1 &= 0 (0 + 10)^* (1 + \eps) + \eps
\end{align*}

Наименьшая неподвижная точка:

$$
\begin{cases}
  Q_0 = (0 + 10)^* (1 + \eps) \\
  Q_1 = 0 (0 + 10)^* (1 + \eps) + \eps \\
  Q_2 = \varnothing
\end{cases}
$$

\smallskip

{\bf (iii)}
Подставим в регулярное выражение $Q_0 = (0 + 10)^* (1 + \eps)$ вместо литералов регулярные выражения для языков $L_0$ и $L_1$ соответственно: $0 \rightarrow a^*$, $1 \rightarrow aba$. Получим регулярное выражение $R = (a^* + aba(a)^*)^* (aba + \eps)$.

Воспользуемся алгоритмом 3.3.3 из книги Серебрякова для построения автомата по $R$. Сначала дополним регулярное выражение символом $\#$, получим $(a^* + aba(a)^*)^* (aba + \eps) \#$. Теперь построим синтаксическое дерево по полученному регулярному выражению.

На диаграмме ниже изображено построенное синтаксическое дерево с результатом вычисления функций $firstpos$ и $lastpos$, значение которых записаны соответственно слева и справа от каждого узла дерева.

\centerline{\includegraphics{{5.2}.png}}

Теперь вычислим значения функции $followpos$. Результаты записаны в следующую таблицу:

\smallskip

\noindent
\centerline{
\begin{tabular}{c|c}
  позиция & followpos \\
  \hline
  1 & $\{ 2 \}$ \\
  2 & $\{ 3 \}$ \\
  3 & $\{ 1, 4, 5, 8, 9 \}$ \\
  4 & $\{ 1, 4, 5, 8, 9 \}$ \\
  5 & $\{ 1, 5, 8, 9 \}$ \\
  6 & $\{ 7 \}$ \\
  7 & $\{ 9 \}$ \\
  8 & $\{ 6 \}$ \\
  9 & $\varnothing$ \\
\end{tabular}
}

\smallskip

По полученным значениям $followpos$ построим автомат.

\centerline{\includegraphics{{5.3}.png}}

На диаграмме выше, изображен автомат, уже дополненный до полного состоянием $q_5$. При этом каждому из состояний $q_0$, $q_1$, $q_2$, $q_3$, $q_4$ соответствует множество позиций синтаксического дерева, указанное в следующей таблице:

\smallskip

\noindent
\centerline{
\begin{tabular}{c|c}
  состояние & множество позиций \\
  \hline
  $q_0$ & $\{ 1, 5, 8, 9 \}$ \\
  $q_1$ & $\{ 1, 2, 5, 6, 8, 9 \}$ \\
  $q_2$ & $\{ 3, 7 \}$ \\
  $q_3$ & $\{ 1, 2, 4, 5, 6, 8, 9 \}$ \\
  $q_4$ & $\{ 1, 4, 5, 8, 9 \}$ \\
\end{tabular}
}

\smallskip

{\bf (iv)}
По алгоритму, описанному в теории построим минимальный эквивалентный автомат $min(A)$. В исходном автомате нет недостижимых состояний, в чем легко убедиться. Теперь построим индуктивное отношение эквивалентности $R \equiv^{|Q| - 2}$, где $|Q| - 2 = 6 - 2 = 4$.
\begin{align*}
  q_0 &\equiv^0 q_1 \equiv^0 q_3 \equiv^0 q_4; \;\; q_2 \equiv^0 q_5; \\
  q_0 &\equiv^1 q_1 \equiv^1 q_3 \equiv^1 q_4; \;\; q_2; \;\; q_5; \\
  q_0 &\equiv^2 q_3; \;\; q_1 \equiv^2 q_4; \;\; q_2; \;\; q_5; \\
  q_0 &\equiv^3 q_3; \;\; q_1 \equiv^3 q_4; \;\; q_2; \;\; q_5; \\
  q_0 &\equiv^4 q_3; \;\; q_1 \equiv^4 q_4; \;\; q_2; \;\; q_5;
\end{align*}

Теперь объединим эквивалентные состояния и построим $min(A)$.

\centerline{\includegraphics{{5.4}.png}}

\smallskip

{\bf (v)}
Для каждого состояния $min(A)$ построим достигающие цепочки.

\smallskip

\noindent
\centerline{
\begin{tabular}{c|c}
  состояние & достигающая цепочка \\
  \hline
  $q_{0, 3}$ & $\eps$ \\
  $q_{1, 4}$ & $a$ \\
  $q_2$ & $ab$ \\
  $q_5$ & $abb$ \\
\end{tabular}
}

\smallskip

Для каждой пары различных состояний построим различающие цепочки.

\smallskip

\noindent
\centerline{
\begin{tabular}{|c|c|c|c|c|}
  \hline
  & $q_{0, 3}$ & $q_{1, 4}$ & $q_2$ & $q_5$ \\
  \hline
  $q_{0, 3}$ & X & $ba$ & $\eps$ & $\eps$ \\
  \hline
  $q_{1, 4}$ & $ba$ & X & $\eps$ & $\eps$ \\
  \hline
  $q_2$ & $\eps$ & $\eps$ & X & $a$ \\
  \hline
  $q_5$ & $\eps$ & $\eps$ & $a$ & X \\
  \hline
\end{tabular}
}

\end{document}
