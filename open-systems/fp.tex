\documentclass[10pt]{article}

\usepackage[utf8]{inputenc}
\usepackage[T2A]{fontenc}
\usepackage[english, russian]{babel}

\usepackage{amssymb, amsmath, textcomp, tabularx, graphicx}
\usepackage{indentfirst}
\usepackage{listings}

\title{Функциональный профиль для beActive}
\author{Андрей Коновалов, 073}
\date{}

\begin{document}

\maketitle

\section{beActive}

beActive - это менеджер личного расписания.
Пользователь может создать аккаунт и манипулировать собственным расписанием, например добавлять или удалять события.
Расписание пользователей хранится на централизованном сервере.

Сервер осуществляет две функции: реализует API для работы с расписанием пользователия и предоставляет пользователю стандартный веб-клиент для работы с этим API (веб-сайт).
У сторонних разработчиков должна быть воможность реализовать собственный клиент (например мобильное приложение) для взаимодействия с сервером.
Это требование порождает необходимость открытости и доступности API для взаимодействия с сервером. 

Требования к открытости системы:
\begin{itemize}
  \item открытые форматы взамодействия клиента с сервером;
  \item переносимость сервера;
  \item легкая управляемость сервера.
\end{itemize}

\section{Функциональный профиль}

В качестве модели для функционального профиля была выбрана модель MUSIC/CCITT.

\subsection{Внешняя среда}

Между сервером и клиентами требуется сетевое соединение (описано в разделе коммуникации).

\subsection{Платформа}

Для запуска серверной части системы требуется операционная система, совместимая с
\begin{itemize}
  \item интерпретатором языка Python;
  \item стандартом базы данных, описанном в разделе базы данных и информация;
  \item стандартом сетевого соединения, описанном в разделе коммуникации.
\end{itemize}

Для запуска клиентской части системы требуется операционная система, совместимая с
\begin{itemize}
  \item веб-браузером, совместимым со стандартами, описанными в разделе пользовательский интерфейс;
  \item стандартом сетевого соединения, описанном в разделе коммуникации.
\end{itemize}

\subsection{Информация}

Вся пользовательская информация хранится на сервере.
Для хранения информации используются базы данных поддерживающие открытый стандарт SQL:2011 (ISO/IEC 9075:2011), совместимые с реализацией операционной системы платформы.

\subsection{Коммуникация}

Для коммуникации между сервером и клиентами используется TLS соединение (стандарт описан в RFC 5246).
Для шифрования передаваемых данных используются алгоритмы, совместимые со стандартом TLS соединения.
Данные по TLS соединению передаются с использованием протокола HTTP (описан в RFC 7230, 7231, 7232, 7233, 7234, 7235), в соответствии со стандартом HTTPS (HTTP over TLS, описан в RFC 2818).

Для взаимодействия клиента с API сервера, используется формат JSON (RFC 7159) поверх HTTPS.

Для коммуникации серверного приложения с базой данных используется стандарт, описанный в реализации базы данных.

\subsection{Пользовательский интерфейс}

В качестве стандартного пользовательского интерфейса используется веб-сайт, использующий HTML (RFC 2854), CSS (RFC 2318) и JavaScript (RFC 4329).
Сторонним разработчикам предоставляется возможность реализовать собственный интерфейс для взаимодействия с сервером, никаких ограничений на который не накладывается.

\subsection{Управление}

Контроль сервера осуществляется вручную администратором посредством инструментов операционной системы (стандарт IEEE Std 1003.2-1992).

Авторизация пользователей осуществляется с использованием стандарта OAuth 2.0.

\section{Реализация}

Для запуска серверной части системы изпользуется операционная система Ubuntu 14.04.
В качестве интерпретатора изпользуется CPython, а в качестве базы данных - MySQL 5.6.
Серверное приложение реализовано на языке Python, с использованием фрейворка Django.
HTTPS соединение реализуется с помощью библиотеки OpenSSL, как часть веб сервера Nginx.

Для взаимодействия со стандартным пользовательским интерфейсом используется любой из современных браузеров (Chrome, Firefox, ...).


\end{document}
